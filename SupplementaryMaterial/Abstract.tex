% Place abstract below.

Recent advances in rotorcraft design, multi-rotor vehicle control, miniaturization of hardware, sensing, and battery technologies have enabled cheap, practical design of micro air vehicles (MAVs) for civilian and hobby applications. In parallel, several applications are being envisioned that bring together networks of MAVs to accomplish large tasks by coordinating with each other. Despite these advancements, and new FAA rules governing their use, it is still very challenging to experiment with multiple networked MAVs. To mitigate these challenges, in this dissertation, we develop an open software/hardware platform called the University at Buffalo's Airborne Networking and Communications Testbed (UB-ANC). The UB-ANC ecosystem comprises three open-source projects: \textbf{UB-ANC Drone}, \textbf{UB-ANC Emulator}, and \textbf{UB-ANC Planner}. Our goal is to design, implement, and test MAV networking applications in simulation, and provide seamless transition to deployment.

\textbf{UB-ANC Drone} provides a flexible drone platform for robotics and network researchers to test and evaluate different mission planning algorithms and networking protocols on actual drones. \textbf{UB-ANC Emulator} provides a virtual environment for researchers to evaluate different algorithms in software and seamlessly transfer them to actual hardware (UB-ANC Drone). Finally, using these two platforms, we developed \textbf{UB-ANC Planner}, which is an energy-efficient coverage path planner that aims to minimize the maximum energy consumption across drones as they search an arbitrary area with obstacles. In this dissertation, we describe all of these projects in detail.