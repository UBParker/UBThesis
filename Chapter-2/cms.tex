\vspace{-3pt}
\section{The CMS Detector}\label{sec:CMSDetector}



The Compact Muon Solenoid (CMS) detector was used to collect the data presented in this thesis, it is one of two large general purpose detectors at the LHC. CMS experiment has recorded 162 $fb^{-1}$ integrated luminosity in the dataset presented in this thesis, collected during Run 2 of LHC.

The Compact Muon Solenoid, CMS, is one of 4 detectors that measure collisions of protons and lead ions produced by the Large Hadron Collider, LHC, at CERN. CMS is the smaller of the 2 large general-purpose detectors, the other being ATLAS. The most notable feature of the detector is it's powerful 3.8 Tesla solenoid magnet, the largest superconducting magnet ever built, as of the year 2011.


The central feature of the CMS apparatus is a superconducting solenoid of 6 m internal diameter, providing a magnetic field of 3.8 T. Within the solenoid volume are a silicon pixel and strip tracker, a lead tungstate crystal electromagnetic calorimeter, and a brass and scintillator hadron calorimeter, each composed of a barrel and two endcap sections. Forward calorimeters, made of steel and quartz-fibres, extend the pseudorapidity coverage provided by the barrel and endcap detectors. Muons are detected in gas-ionization chambers embedded in the steel flux-return yoke outside the solenoid. 

%https://twiki.cern.ch/twiki/bin/viewauth/CMS/Internal/PubDetector

