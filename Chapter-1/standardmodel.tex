%%%%%%%%%%%%%%%%%%%%%%%%%%%%%%%%%%%%%%%%%%%%%%%%%%%%%%
\section{Introduction To The Standard Model}\label{secSM}



The Standard Model (SM) of particle physics is a quantum field theory (QFT) description of the strong, weak and electromagnetic forces of nature. The known particles of the SM are; 1 scalar Higgs boson,4 guage bosons, 6 types of quark, and 6 types of lepton. 

The quarks and leptons, particles which constitute matter, are fermions, obeying Fermi-Dirac statistics due to their half-integer spin. In contrast,the bosons have integer spin and obey Bose-Einstein statistics. Gauge bosons mediate the 3 fundamental forces and the Higg's boson is responsible for the electro-weak symmetry breaking which gives mass to the other particles\cite{Griffiths:111880}. 

The fermions are arranged into 3 generations, arranged in columns from left to right on \cite{fig:SM}



Quantum Chromodynamics, QCD, is the theory of the strong interaction which governs the interactions of quarks and gluons\cite{Griffiths:111880}.


The SM constitues humanity's most rigorous theory of our universe, providing predictions of observables which have since been measured, in the case of Quantum Electrodynamics, QED, to the highest precision of any scientific theory. Despite the impressive predictions,the gravitational force and more subtle phenomena, such as flavor oscillation of neutrinos \cite{Ashie:2005ik}, indicate the existence of physics beyond the standard model, BSM.

Various attempts have been made to unify the fundamental forces under one theory, thusfar the electromagnetic and weak interactions have been united by electro-weak theory. 

The Standard electroweak model can be described $SU(2) x U(1)$ mathematically.

The  $SU(2) x U(1)$ guage group is a convolution ( $<- $That is not the right word...) of the special unitary symmetry group $SU(2)$ describing 3 mixed massive vector bosons, $W_{-}$ $W_{+}$ $Z_0$, carriers of the weak nuclear force and the unitary gauge group $U(1)$ , describing the lonely massless chargeless photon, of the electromagnetic interaction.

The standard model of the strong interaction is known as quatum chromodynamics, QCD, a non-Abelian guage theory described by the special unitary group $(SU(3)_f)$, where the  flavours of quark are the physical manifestation of the symmetry group. This force is mediated by the 8 massless gluons which carry color charge, making QCD more complicated mathematically than QED.

The SM also contains a Higgs boson, an excitation of a scalar Higg's field, which gives rise to spantaneous symmetry breaking of the electroweak theory, providing the particles with mass, but I won't get into that. 

The quarks and leptons are arranged in generations according to their relative masses, as shown in Figure \ref{fig:SM}. The table also shows the spins of the particles, the leptons and quarks have half-integer spin, fermions, that obey the fermi exclusion principle, conversely the bosons have half integer spin and therefore obey bose-einstein statitics. Through the SM we interpret the observed hadronic particles, mesons ( baryons ) , as 2 quark (3 quark) bound states. The existence of spin $\frac{3}{2}$ baryons, which are symmetric bound states in space, spin and flavour and the need to obey Fermi-Dirac statistics, by maintaining total assymmetry of the wavefunction,implies there is another degree of freedom, called color, so that each quark is either red, green or blue. Granted only color singlet, containing either all 3 or 1 and it's anti color, states exist. Furthermore there exists a property of asymptotic freedom where the QCD coupling between quarks and gluons increases as they asymptotically approach one another. There exists a wealth of experiemental data to support the concept of asymptotic freedom despite the fact that rigorous mathematical proof of the exlusion of free quark and gluon states has yet to be acheived.


%image CMS:2014mna
% visuals/strong-coupling-cms


\begin{figure}[htb]
\centering
\includegraphics[width=1.0\textwidth]{visuals/strong-coupling-cms2.png}
\caption{The running of the strong coupling constand as compiled by CMS including measurements from CMS and HERA among others~\cite{CMS:2014mna}.}
\label{fig:alphas}
\end{figure}





%another DY thesis  http://inspirehep.net/record/1345977/files/DoolingSamantha_Dissertation.pdf

Assymptotic freedom is a useful property as it allows for perturbative calculations of QCD observables, this is discussed in section XXX.

% Symmetris imply conserved quantities, Neuther's Theorem

 Nuclei in ordinary matter are composed solely of $1^{st}$ generation particles, up and down quarks, bound by gluons. Neutral atoms contain an equal number of protons (composed of 2 up quarks and a down quark) and electrons, $1^{st}$ generation leptons. The main distinction between leptons and quarks, both fermions (particles of $\frac{1}{2}$ integer spin), being that leptons do not experience the color interaction $(SU(3)_f)$ like their quark friends. In each generation there is a quark with charge $Q = + \frac{2}{3}$ (up, charm, top) and another of charge $Q = - \frac{1}{3}$ (down, strange, bottom).



\begin{figure}[htb]
\centering
\includegraphics[width=1.0\textwidth]{smdiagram.pdf}
\caption{Fundamental particles of the Standard Model~\cite{modellinginvisible}.}
\label{fig:SM}
\end{figure}



\subsection{Quantum Chromodynamics}\label{secQCD}


lagrangian

\begin{equation}
\mathcal{L}=-\frac{1}{4} F_{\mu \nu}^{A} F_{A}^{\mu \nu}+\sum_{\text {flavours }} \overline{\psi}_{a}\left(i \gamma_{\mu} D^{\mu}-m\right)_{a b} \psi_{b}
\end{equation}

\begin{equation}
F_{\mu \nu}^{A}=\partial_{\mu} A_{\nu}^{A}-\partial_{\nu} A_{\mu}^{A}+g_{s} f^{A B C} A_{\mu}^{B} A_{\nu}^{C}
\end{equation}

covariant derivative
\begin{equation}
\left(D_{\mu}\right)_{a b}=\partial_{\mu} \delta_{a b}-i g_{s} A_{\mu}^{A} t_{a b}^{A}
\end{equation}










%%%%%%%%%%%% NEW  CHAPTER %%%%%%%%%%%%%%%%%%%%





