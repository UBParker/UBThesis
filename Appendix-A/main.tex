

\chapter{Relativistic Kinematics}\label{chap:RelativisticKinematics}


\begin{figure}[htb]

\centering
\includegraphics[width=1.0\textwidth]{visuals/1to2splitting.png}
\caption{ A simple $ 1 \rightarrow 2  $ decay described by relativistic kinematics.}
\label{onetotwo}
\end{figure}

Consider the LO process depicted in ~\ref{onetotwo}, of a simple $ 1 \rightarrow 2  $ decay.


at LO there is a kinematic turn-off at $\pt R/2$ (where $R$ is the
distance parameter for the jet clustering), from the relativistic kinematics of a $1\rightarrow 2$ decay. 
However, for real jets the turn-off is closer to $\pt R/\sqrt{2}$ due to stochastic effects. 
To see this, consider a particle of energy $E$ and mass $m$ decaying to two massless
particles, each with an energy $E/2$ and separated by an angle $\theta$. 
The mass must satisfy $m^2 < \frac{E^2}{2}\left( 1 - \cos{\theta}\right)$. In the small
angle limit, this would be $m^2 < E^2\theta^2/4$, or $m < E\theta/2$.





According to relativistic kinematics the interaction can be described by the following equation:\newline

$p^{\mu} p_{\mu}  =  (p_1 + p_2)^{\mu} (p_1 + p_2)_{\mu}  $\newline

$p^{\mu} p_{\mu}  =  (p_1 + p_2)^{\mu} (p_1 + p_2)_{\mu}  $\newline

$m^2  = (E_1 + E_2)^2  - (\vec{p_1} + \vec{p_2} ) \dot (\vec{p_1} + \vec{p_2} )  $\newline


$m^2  \simeq 2 E_1 E_2 (1 - \cos \theta ) $\newline

If the energies of the splitting particles are equal then the equation simplifies since  $ E_1 = E_2 = \frac{E}{2} $ .


$m^2  < \frac{E^2}{2} ( 1 - \cos{\theta}  $\newline

$\frac{2m^2}{E^2}  < ( 1 - \cos{\theta}  $\newline

Using the small angle approximation this simplifies further.

$(1 - (1- \frac{\theta^2}{2}) )  \simeq  \frac{\theta^2}{2}  $\newline

Solving for mass, one finds :\newline

$m < E\theta/2$\newline

With more particle decays, the stochastic nature of the shower increases this to $m < E\theta/\sqrt{2}$.
Thus, a leading-order ($1\rightarrow 2$) decay will have a faster kinematic
turn-off than an all-orders ($1 \rightarrow$ many) decay.\newline

Solving for theta:\newline
 $\theta < \frac{2}{\gamma}$ where $\gamma$ is the lorentz factor $\gamma = \frac{1}{\sqrt{1-\frac{v^2}{c^2}}}  $.