% Place abstract below.

The standard model of particle physics, while describing our universe well on many scales, has yet to be precisely measured in all energy regimes. Recent theoretical advances in higher order QCD calculations have provided a way to compare the standard model's predictions to precision measurements of data and simulation. Within this dissertation, I present a measurement of the double differential jet production cross section as a function of the jet mass and transverse momentum. The measurement is presented in events with a Z + Jet topology, with and without soft radiation included in the jet. The removal of the soft radiation is done using a jet "grooming" algorithm.

% rivello: I guess this line should have been removed:
 a jet grooming algorithm applied.

Studying Z + jet events will yield a light quark enriched jet sample, which has not yet been studied at $\sqrt{s}$ = 13 TeV. Comparing groomed and ungroomed jets will allow us to better understand the jet mass in all energy regimes since the groomed jets will have varying amounts of soft and collinear radiation with respect to the ungroomed counterpart. For ungroomed jets, leading-order and next-to-leading order QCD Monte Carlo programs are found to predict the jet mass spectrum in the data reasonably well, with some disagreement at very low masses. For groomed jets, the agreement between the Monte Carlo programs and the data improves overall, and extends lower in jet mass due to the removal of soft and colinear portions of the jet. First-principles theoretical calculations of the groomed jet mass are being prepared currently by collegues in order to make comparisons with the measurement presented herein. Ultimately these measurements will be used to tune Monte Carlo generators, producing more accurate parton showering simulations, leading to tighter constraint of backgrounds in future searches for new physics.
