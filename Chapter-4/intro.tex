\section{Introduction}
\label{sec:ch4:intro}
As noted previously, networked unmanned~~aerial~~vehicles~~(UAVs)~~have emerged as an important technology for public safety, commercial, and military applications including search and rescue, disaster relief, precision agriculture, environmental monitoring, and surveillance. Many of these applications require sophisticated mission planning algorithms to coordinate multiple drones to cover an area efficiently. Such scenarios are complicated by the existence of obstacles, such as buildings, requiring detailed planning for effective operation. Although a lot of work has been done on mission planning, optimal mission planning solutions depend heavily on the specific types of vehicles considered (e.g., ground robots, indoor drones, or outdoor drones), their kinematics, and the specific applications. Prior techniques have been optimized for shortest time to completion or control efficiency. However, a major challenge in the realization of such solutions is the limited energy on each drone. 

