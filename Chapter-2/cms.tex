\vspace{-3pt}
\section{The CMS Detector}\label{sec:CMSDetector}



The Compact Muon Solenoid (CMS) detector was used to collect the data presented in this thesis, it is one of two large general purpose detectors at the LHC. CMS experiment has recorded 162 $fb^{-1}$ integrated luminosity in the dataset presented in this thesis, collected during Run 2 of LHC.

CMS is one of 4 detectors that measure collisions of protons and lead ions produced by the Large Hadron Collider, LHC, at CERN. CMS is the smaller, overall length of 22m, a diameter of 15m, and weighs 14\,000 tonnes, of the 2 large general-purpose detectors, the other being ATLAS. The most notable feature of the detector is it's powerful 3.8 Tesla solenoid magnet, the largest superconducting magnet ever built, as of the year 2011.


The central feature of the CMS apparatus is a superconducting solenoid of 6 m internal diameter, providing a magnetic field of 3.8 T. Within the solenoid volume are a silicon pixel and strip tracker, a lead tungstate crystal electromagnetic calorimeter, and a brass and scintillator hadron calorimeter, each composed of a barrel and two endcap sections. Forward calorimeters, made of steel and quartz-fibres, extend the pseudorapidity coverage provided by the barrel and endcap detectors. Muons are detected in gas-ionization chambers embedded in the steel flux-return yoke outside the solenoid. 


%https://twiki.cern.ch/twiki/bin/viewauth/CMS/Internal/PubDetector

\subsection{Calorimeter Energy Resolution}\label{sec:CMSDetectorCalo}

 In the barrel section of the ECAL, an energy resolution of about 1\% is achieved for unconverted or late-converting photons that have energies in the range of tens of GeV. The remaining barrel photons have a resolution of about 1.3\% up to a pseudorapidity of $\abs{\eta} = 1$, rising to about 2.5\% at $\abs{\eta} = 1.4$. In the endcaps, the resolution of unconverted or late-converting photons is about 2.5\%, while the remaining endcap photons have a resolution between 3 and 4\%~\cite{CMS:EGM-14-001}.The lead tungstate crystals are $25.8 X_0$ thick in the barrel and $24.7 X_0$ thick in the endcaps. When combining information from the entire detector, the jet energy resolution amounts typically to 15\% at 10\GeV, 8\% at 100\GeV, and 4\% at 1\TeV, to be compared to about 40\%, 12\%, and 5\% obtained when the ECAL and HCAL calorimeters alone are used.

 \section{From Calorimeter Energy Deposits to Jets}\label{sec:CMSDetectorJets}

In the region $\abs{\eta} < 1.74$, the HCAL cells have widths of 0.087 in pseudorapidity and 0.087 in azimuth ($\phi$). In the $\eta$-$\phi$ plane, and for $\abs{\eta} < 1.48$, the HCAL cells map on to $5 \times 5$ arrays of ECAL crystals to form calorimeter towers projecting radially outwards from close to the nominal interaction point. For $\abs{ eta }$ > 1.74, the coverage of the towers increases progressively to a maximum of 0.174 in $\Delta \eta$ and $\Delta \phi$. Within each tower, the energy deposits in ECAL and HCAL cells are summed to define the calorimeter tower energies, subsequently used to provide the energies and directions of hadronic jets.

Jets are reconstructed offline from the energy deposits in the calorimeter towers, clustered using the anti-\kt algorithm~\cite{Cacciari:2008gp, Cacciari:2011ma} with a distance parameter of 0.4. In this process, the contribution from each calorimeter tower is assigned a momentum, the absolute value and the direction of which are given by the energy measured in the tower, and the coordinates of the tower. The raw jet energy is obtained from the sum of the tower energies, and the raw jet momentum by the vectorial sum of the tower momenta, which results in a nonzero jet mass. The raw jet energies are then corrected to establish a relative uniform response of the calorimeter in $\eta$ and a calibrated absolute response in transverse momentum \pt. 


 \section{Muon Reconstruction}\label{sec:CMSDetectorMuons}


Muons are measured in the pseudorapidity range $\abs{\eta} < 2.4$, with detection planes made using three technologies: drift tubes, cathode strip chambers, and resistive plate chambers. The single muon trigger efficiency exceeds 90\% over the full $\eta$ range, and the efficiency to reconstruct and identify muons is greater than 96\%. Matching muons to tracks measured in the silicon tracker results in a relative transverse momentum resolution, for muons with \pt up to 100\GeV, of 1\% in the barrel and 3\% in the endcaps. The \pt resolution in the barrel is better than 7\% for muons with \pt up to 1\TeV~\cite{Sirunyan:2018fpa}. 


A more detailed description of the CMS detector, together with a definition of the coordinate system used and the relevant kinematic variables, can be found in Ref.~\cite{Chatrchyan:2008zzk}.  The global event reconstruction (also called particle-flow event reconstruction~\cite{CMS-PRF-14-001}) is described in Section~\ref{sec:PFReco}.


