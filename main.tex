%%%%%%%%%%%%%%%%%%%%%%%%%%%%%%%%%%%%%%%%%%%%%%%%%%%%%%%%%%%%%%%
%
%     filename  = "Dissertation Index.tex",
%     version   = "Draft 1",
%     date      = "1/16/2013",
%     authors   = "Nicholas P. Nicoletti",
%     copyright = "Nicholas P. Nicoletti",
%     address   = "Department of Political Science,
%                  516 Park Hall,
%                  University at Buffalo,
%                  Buffalo, NY 14260,
%                  USA",
%     telephone = "(585) 752-5167",
%     email     = "npn@buffalo.edu",
%
%%%%%%%%%%%%%%%%%%%%%%%%%%%%%%%%%%%%%%%%%%%%%%%%%%%%%%%%%%%%%%%
%
% Change History:
%
% Draft Version 1.0 - No Changes.
%
%%%%%%%%%%%%%%%%%%%%%%%%%%%%%%%%%%%%%%%%%%%%%%%%%%%%%%%%%%%%%%%
%
% This is a template file to help get you started using the
% psuthesis.cls for theses and dissertations at Penn State
% University. You will, of course, need to put the
% psuthesis.cls file someplace that LaTeX will find it.
%
% We have set up a directory structure that we find to be clean
% and convenient. You can readjust it to suit your tastes. In
% fact, the structure used by our students is even a little
% more involved and commands are defined to point to the
% various directories.
%
% This document has been set up to be typeset using pdflatex.
% About the only thing you will need to change if typesetting
% using latex is the \DeclareGraphicsExtensions command.
%
% The psuthesis document class uses the same options as the
% book class. In addition, it requires that you have the
% ifthen, calc, setspace, and tocloft packages.
%
% The first additional option specifies the degree type. You
% can choose from:
%     Ph.D. using class option <phd>
%     M.S. using class option <ms>
%     M.Eng. using class option <meng>
%     M.A. using class option <ma>
%     B.S. using class option <bs>
%     B.A. using class option <ba>
%     Honors Baccalaureate using the option <honors>
%
% If you specify either ba or bs in addition to honors, it will
% just use the honors option and ignore the ba or bs option.
%
% The second additional option <inlinechaptertoc> determines
% the formatting of the Chapter entries in the Table of
% Contents. The default sets them as two-line entries (try it).
% If you want them as one-line entries, issue the
% inlinechaptertoc option.
%
% The class option ``honors'' should be used for theses
% submitted to the Schreyer Honors College. This option
% changes the formatting on the Title page so that the
% signatures appear on the Title page. Be sure and comment
% out the command \psusigpage when using this option since it
% is not needed and it messes up the vertical spacing on the
% Title page.
%
% The class option ``honorsdepthead'' adds the signature of the
% department head on the Title page for those baccalaureate
% theses that require this.
%
% The class option ``secondthesissupervisor'' should be used
% for baccalaureate honors degrees if you have a second
% Thesis Supervisor.
%
% The vita is only included with the phd option and it is
% placed at the end of the thesis. The permissions page is only
% included with the ms, meng, and ma options.
%%%%%%%%%%%%%%%%%%%%%%%%%%%%%%%%%%%%%%%%%%%%%%%%%%%%%%%%%%%%%%%
% Only one of the following lines should be used at a time.
\documentclass[phd,12pt]{psuthesis}
%\documentclass[draft,phd,inlinechaptertoc]{psuthesis}
%\documentclass[draft,ms]{psuthesis}
%\documentclass[draft,honorsdepthead,honors]{psuthesis}
%\documentclass[draft,honors]{psuthesis}
%\documentclass[draft,secondthesissupervisor,honors]{psuthesis}
%\documentclass[draft,bs]{psuthesis}


%%%%%%%%%%%%%%%%%%%%%%%%%%%%
% Packages we like to use. %
%%%%%%%%%%%%%%%%%%%%%%%%%%%%
\usepackage{amsmath}
\usepackage{amssymb}
\usepackage{amsthm}
\usepackage{exscale}
\usepackage[mathscr]{eucal}
\usepackage{bm}
\usepackage{eqlist} % Makes for a nice list of symbols.
\usepackage[final]{graphicx}
\usepackage[dvipsnames]{color}
\DeclareGraphicsExtensions{.pdf, .jpg, .png}
\usepackage{natbib}
\usepackage{har2nat}
\usepackage{verbatim}
\usepackage{url}
\usepackage{longtable}
\usepackage{mathpazo}
\usepackage{pstricks}
\usepackage{sgamevar}
\usepackage{egameps}
\def\citeapos#1{\citeauthor{#1}'s \citeyear{#1}}
\newenvironment{my_enumerate}
{\begin{enumerate}
  \setlength{\itemsep}{1pt}
  \setlength{\parskip}{0pt}
  \setlength{\parsep}{0pt}}{\end{enumerate}}
\newenvironment{my_itemize}
{\begin{itemize}
  \setlength{\itemsep}{1pt}
  \setlength{\parskip}{0pt}
  \setlength{\parsep}{0pt}}{\end{itemize}}


%%%%%%%%%%%%%%%%%%%%%%%%
% Setting for fncychap %
%%%%%%%%%%%%%%%%%%%%%%%%
% Comment out or remove the next two lines and you will get
% the standard LaTeX chapter titles. We like these A LOT
% better.
\usepackage[Lenny]{fncychap}
\ChTitleVar{\Huge\sffamily\bfseries}


%%%%%%%%%%%%%%%%%%%%%%%%%%%%%%%
% Use of the hyperref package %
%%%%%%%%%%%%%%%%%%%%%%%%%%%%%%%
%
% This is optional and is included only for those students
% who want to use it.
%
% To use the hyperref package, uncomment the following line:
%\usepackage[colorlinks=true,urlcolor=purple,citecolor=green,linkcolor=blue]{hyperref}
%
% Note that you should also uncomment the following line:
%\renewcommand{\theHchapter}{\thepart.\thechapter}
%
% to work around some problem hyperref has with the fact
% the psuthesis class has unnumbered pages after which page
% counters are reset.


%%%%%%%%%%%%%%%%%%%%%%%%%%%%%%%%%%%%
% SPECIAL SYMBOLS AND NEW COMMANDS %
%%%%%%%%%%%%%%%%%%%%%%%%%%%%%%%%%%%%
% Place user-defined commands below.

\graphicspath{
{Chapter-2/Figures/}
{Chapter-3/Figures/}
{Chapter-4/Figures/}
}

\usepackage{xspace}
\usepackage{mfirstuc}
\usepackage{multirow}
\usepackage{subcaption}

\newcommand{\engine}{Emulation Engine}
\newcommand*{\eg}{e.g.\@\xspace}
\newcommand*{\ie}{i.e.\@\xspace}
\newcommand{\degree}{$^{\circ}$}

\usepackage{setspace}
\doublespacing



%%%%%%%%%%%%%%%%%%%%%%%%%%%%%%%%%%%%%%%%%
% Renewed Float Parameters              %
% (Makes floats fit better on the page) %
%%%%%%%%%%%%%%%%%%%%%%%%%%%%%%%%%%%%%%%%%
\renewcommand{\floatpagefraction}{0.85}
\renewcommand{\topfraction}      {0.85}
\renewcommand{\bottomfraction}   {0.85}
\renewcommand{\textfraction}     {0.15}

% ----------------------------------------------------------- %

%%%%%%%%%%%%%%%%
% FRONT-MATTER %
%%%%%%%%%%%%%%%%
% Title
\title{Measurement of the double differential jet production cross section with respect to jet mass and transverse momentum in Z + Jet events from proton - proton collisions at $\sqrt{s}$ = 13 TeV using the CMS detector at LHC}

% Author and Date of Degree Conferral or Defense
\author{Ashley Marie Parker}
% the degree will be conferred on this date
\degreedate{August 2019}
% year of your copyright. I have removed this from the cover page because UB's guidelines do not include it.
%\copyrightyear{2019}

% This is the document type. For example, this could also be:
%     Comprehensive Document
%     Thesis Proposal
\documenttype{Disseration}
%The department where you will be submitting the document%
\dept{Department of Physics}
% This will generally be The Graduate School, though you can
% put anything in here to suit your needs. This has also been removes from the cover page via the psuthesis.cls document because UB guidelines do not allow for it.
\submittedto{The Graduate School}


%%%%%%%%%%%%%%%%%%
% Signatory Page %
%%%%%%%%%%%%%%%%%%
% You can have up to 7 committee members, i.e., one advisor
% and up to 6 readers.
%
% Begin by specifying the number of readers.
\numberofreaders{3}


% Input reader information below. The optional argument, which
% comes first, goes on the second line before the name.
\advisor[Thesis Advisor, Chair of Committee]
        {Salvatore Rappoccio}
        {Professor of Physics}

\readerone[Committee Member]
          {Ia Iashvili}
          {Professor of  Physics}

\readertwo[Committee Member]
          {Ciaran Williams}
          {Professor of  Physics}




% Makes use of LaTeX's include facility. Add as many chapters
% and appendices as you like.
\includeonly{%
Chapter-1/main,%
Chapter-2/main,%
Chapter-3/main,%
Chapter-4/main,%
Chapter-5/main,%
Appendix-A/main,%
Appendix-B/main%
}

%%%%%%%%%%%%%%%%%
% THE BEGINNING %
%%%%%%%%%%%%%%%%%
\begin{document}

%%%%%%%%%%%%%%%%%%%%%%%%
% Preliminary Material %
%%%%%%%%%%%%%%%%%%%%%%%%
% This command is needed to properly set up the frontmatter.
\frontmatter

%%%%%%%%%%%%%%%%%%%%%%%%%%%%%%%%%%%%%%%%%%%%%%%%%%%%%%%%%%%%%%
% IMPORTANT
%
% The following commands allow you to include all the
% frontmatter in your thesis. If you don't need one or more of
% these items, you can comment it out. Most of these items are
% actually required by the Grad School -- see the Thesis Guide
% for details regarding what is and what is not required for
% your particular degree.
%%%%%%%%%%%%%%%%%%%%%%%%%%%%%%%%%%%%%%%%%%%%%%%%%%%%%%%%%%%%%%
% !!! DO NOT CHANGE THE SEQUENCE OF THESE ITEMS !!!
%%%%%%%%%%%%%%%%%%%%%%%%%%%%%%%%%%%%%%%%%%%%%%%%%%%%%%%%%%%%%%

% Generates the signature page. This is not bound with your
% thesis.
%\psusigpage

\pagestyle{plain}
\pagenumbering{roman}

% Generates the title page based on info you have provided
% above.
\psutitlepage

%Generates Copyright Page
\copyrightpage{SupplementaryMaterial/Copyright}

\newpage
% Generates the committee page -- this is bound with your
% thesis. If this is an baccalaureate honors thesis, then
% comment out this line.
% \psucommitteepage

% Generates the Epigraph/Dedication. The first argument should
% point to the file containing your Epigraph/Dedication and
% the second argument should be the title of this page.
%\thesisdedication{SupplementaryMaterial/Dedication}{Dedication}

% Generates the Acknowledgments. The argument should point to
% the file containing your Acknowledgments.
%\thesisacknowledgments{SupplementaryMaterial/Acknowledgments}

% Generates the Table of Contents
\thesistableofcontents

% Generates the List of Tables
\thesislistoftables

% Generates the List of Figures
\thesislistoffigures

% Generates the List of Symbols. The argument should point to
% the file containing your List of Symbols.
% \thesislistofsymbols{SupplementaryMaterial/ListOfSymbols}

% Generates the abstract. The argument should point to the
% file containing your abstract.
\thesisabstract{SupplementaryMaterial/Abstract}


\pagenumbering{arabic}

%%%%%%%%%%%%%%%%%%%%%%%%%%%%%%%%%%%%%%%%%%%%%%%%%%%%%%
% This command is needed to get the main part of the %
% document going.                                    %
%%%%%%%%%%%%%%%%%%%%%%%%%%%%%%%%%%%%%%%%%%%%%%%%%%%%%%
\thesismainmatter

%%%%%%%%%%%%%%%%%%%%%%%%%%%%%%%%%%%%%%%%%%%%%%%%%%
% This is an AMS-LaTeX command to allow breaking %
% of displayed equations across pages. Note the  %
% closing the "}" just before the bibliography.  %
%%%%%%%%%%%%%%%%%%%%%%%%%%%%%%%%%%%%%%%%%%%%%%%%%%
\allowdisplaybreaks{
%
%%%%%%%%%%%%%%%%%%%%%%
% THE ACTUAL CONTENT %
%%%%%%%%%%%%%%%%%%%%%%
% Chapters
\chapter{Introduction}\label{chap:intro}

% \vspace{-5pt}
\section{Motivation}\label{sec:ch1:intro}

Within this dissertation, I provide a measurement of the differential jet cross section, as a function of the jet mass and transverse momentum, in events with a Z + Jet topology, with and without a jet grooming algorithm applied, using data collected by CMS experiment at LHC. The jet grooming used was the ``Soft Drop''
procedure 
%~\cite{softdrop}, wuth multiple values of the tunable parameters $\beta$ and $z_{cut}$. This represents the first, to my knowledge, measurement of it's kind with a light quark enriched jet sample at $\sqrt{s}$ = 13 TeV. 

Softdrop iteratively declusters a jet $j$ with distance parameter $R$ into two subjets, $j_1$ and $j_2$.
If the softdrop condition

\begin{equation}
  \frac{\min(p_{T1},p_{T2})}{p_{T1}+p_{T2}} > z_{cut} \cdot (\frac{\Delta R_{12}}{R})^\beta
\end{equation}

is met, then the procedure stops and $j$ is the final jet. Otherwise, the declustering continues - 
the higher $pt$ subjet is relabeled as $j$ and the lower $pt$ one is dropped.
By design, this condition fails for wide-angle soft radiation, which is therefore removed by the soft
drop procedure. The tunable parameters, $\beta$ and $z_{cut}$, control the degree of jet grooming:
$\beta$ tunes the algorithm's sensitivity to wide-angle radiation, while $z_{cut}$ sets the energy scale
of the grooming. In the case of $\beta \rightarrow \infty$, an ungroomed jet is returned. 
In the $\beta = 0$ case, the soft drop procedure is identical to the ``modified mass drop tagger'' (MMDT)
from Ref.
%~\cite{mmdt}. The soft drop algorithm removes soft and wide-angle radiation
from jets in a very theoretically controlled manner, making it suitable to separate
the ``hard'' and ``soft'' parts of the jet. Specifically, the soft drop
algorithm can remove non-global logarithms from correlations of
radiation within and between jets, which are extremely difficult to
compute theoretically
%~\cite{Dasgupta:2001sh,mmdt,softdrop,Dasgupta:2013via,Dasgupta:2015yua,Larkoski:2015zka}.

Comparing the production cross section for groomed and ungroomed jets separately allows us to
gain sensitivity to both the ``hard'' and ``soft'' jet physics. 
The groomed cross section can be directly compared to theoretical calculations of the jet mass
now and in the future, which is a very active area of theoretical research
at this time
%~\cite{Dasgupta:2012hg,Chien:2012ur,Jouttenus:2013hs,Almeida:2014uva,Liu:2014oog,Stewart:2014nna,Khelifa-Kerfa:2015mma,Frye:2016aiz,Kolodrubetz:2016dzb}. Furthermore, separating the hard and soft jet physics
allows a deeper understanding of the various effects involved in QCD
%radiation. In particular, Ref.~\cite{Frye:2016aiz} calculates the
groomed jet mass at next-to-next-to-leading order using soft colinear effective theory, matched to a
%parton shower at leading order using {\tt MCFM}~\cite{MCFM1,MCFM2}, and the authors of Ref.~\cite{mmdt} have
provided a next-to-leading logarithm calculation with traditional perturbative QCD, matched to a 
parton shower at leading order, also using {tt MCFM}.  We compare these theoretical predictions 
to our data in this paper for the first time in this channel at CMS. Both CMS and ATLAS have similar measurements in a dijet sample at %Ref.~\cite{cms_jetmassDijet, atlas_jetmass2}.

The analysis strategy is similar to that of %Ref.~\cite{cms_jetmassDijet}.
However, there are several differences. As in that paper, the cross section is also
unfolded in both jet mass and $pt$, however we also provide the measurement in jet $\rho$, dimensionless mass , and $pt$. While the previous measurement considered only one value for the soft drop parameter $\beta$, this analysis considers several.
We apply the soft drop algorithm to compare
directly to theoretical computations. Additionally, we not only measure the cross section as a function of mass, but also as a function of dimensionless mass, $\rho = 2log(m/(pt R))$, as is also done in the previously mentioned ATLAS measurement.  The dimensionless mass $\rho$ only weakly depends on $pt$, unlike mass, which is highly correlated. Additionally, the use of this variable aids in the separation of fixed order, perturbative and non-perturbative effects.
Finally, we also present the normalized differential cross section. We compute the cross sections normalized per $pt$ bin
(the ``normalized'' cross section) with respect to the jet $pt$ and jet mass 
by unfolding a binned two-dimensional distribution in $pt$ and mass
with widths $\Delta pt$ and $\Delta m$, respectively.

The normalized differential cross section in two forms :

\begin{equation}
\frac{1}{d\sigma/dpt}\frac{d^2\sigma}{dpt\,dm} = \frac{1}{N/\Delta pt} R(\frac{N_{ij}}{ \Delta pt \,\Delta m} )
\end{equation}


as well as :

\begin{equation}
\frac{1}{d\sigma/dpt}\frac{d^2\sigma}{dpt\,d\rho} = \frac{1}{N/\Delta pt} R(\frac{N_{ij}}{ \Delta pt \,\Delta \rho} )
\end{equation}


where $N$ is the total number of $Z+$jets events in our selection,
$N_{ij}$ is the number of such events in $pt$ bin $i$ and mass ($\rho$) bin $j$,
and $R(\alpha)$ is the unfolding procedure applied to the two-dimensional
distribution $\alpha$.

The 2 above normalized distributions are provided within for ungroomed  and groomed jets Anti-Kt Radious R$= 0.8$ jets. The groomed measurement is given in 9 configurations, one measurement is shown for jets groomed with every combination of 3 possible $\beta$ and $z_{cut}$ values (Where $\beta$ = 0 and $z_{cut}$ = 0 .1 is the current CMS default ):


$ \beta = [ 1,  0 , -1 ]  $

$ z_{cut}  = [ 0.15, 0.1, 0.05 ] $ 


These measurements currently represent humanity's highest energy measurement of a light quark enriched jet production cross section.


\chapter{CMS Experiment at LHC}\label{chap:CMS}

\vspace{-3pt}
\section{The Large Hadron Collider}\label{sec:ch2:lhc}

The LHC is the largest machine created my mankind to date and currently the world's highest energy particle accelerator, it accelerates and collides bunches of $~10^{11}$ protons at a time which collide at combined center-of-mass energy of 13 TeV.



\vspace{-3pt}
\section{The CMS Detector}\label{sec:CMSDetector}



The Compact Muon Solenoid (CMS) detector was used to collect the data presented in this thesis, it is one of two large general purpose detectors at the LHC. CMS experiment has recorded 162 $fb^{-1}$ integrated luminosity in the dataset presented in this thesis, collected during Run 2 of LHC.

CMS is one of 4 detectors that measure collisions of protons and lead ions produced by the Large Hadron Collider, LHC, at CERN. CMS is the smaller, overall length of 22m, a diameter of 15m, and weighs 14\,000 tonnes, of the 2 large general-purpose detectors, the other being ATLAS. The most notable feature of the detector is it's powerful 3.8 Tesla solenoid magnet, the largest superconducting magnet ever built, as of the year 2011.


The central feature of the CMS apparatus is a superconducting solenoid of 6 m internal diameter, providing a magnetic field of 3.8 T. Within the solenoid volume are a silicon pixel and strip tracker, a lead tungstate crystal electromagnetic calorimeter, and a brass and scintillator hadron calorimeter, each composed of a barrel and two endcap sections. Forward calorimeters, made of steel and quartz-fibres, extend the pseudorapidity coverage provided by the barrel and endcap detectors. Muons are detected in gas-ionization chambers embedded in the steel flux-return yoke outside the solenoid. 


%https://twiki.cern.ch/twiki/bin/viewauth/CMS/Internal/PubDetector

\subsection{Calorimeter Energy Resolution}\label{sec:CMSDetectorCalo}

 In the barrel section of the ECAL, an energy resolution of about 1\% is achieved for unconverted or late-converting photons that have energies in the range of tens of GeV. The remaining barrel photons have a resolution of about 1.3\% up to a pseudorapidity of $\abs{\eta} = 1$, rising to about 2.5\% at $\abs{\eta} = 1.4$. In the endcaps, the resolution of unconverted or late-converting photons is about 2.5\%, while the remaining endcap photons have a resolution between 3 and 4\%~\cite{CMS:EGM-14-001}.The lead tungstate crystals are $25.8 X_0$ thick in the barrel and $24.7 X_0$ thick in the endcaps. When combining information from the entire detector, the jet energy resolution amounts typically to 15\% at 10\GeV, 8\% at 100\GeV, and 4\% at 1\TeV, to be compared to about 40\%, 12\%, and 5\% obtained when the ECAL and HCAL calorimeters alone are used.

 \section{From Calorimeter Energy Deposits to Jets}\label{sec:CMSDetectorJets}

In the region $\abs{\eta} < 1.74$, the HCAL cells have widths of 0.087 in pseudorapidity and 0.087 in azimuth ($\phi$). In the $\eta$-$\phi$ plane, and for $\abs{\eta} < 1.48$, the HCAL cells map on to $5 \times 5$ arrays of ECAL crystals to form calorimeter towers projecting radially outwards from close to the nominal interaction point. For $\abs{ eta }$ > 1.74, the coverage of the towers increases progressively to a maximum of 0.174 in $\Delta \eta$ and $\Delta \phi$. Within each tower, the energy deposits in ECAL and HCAL cells are summed to define the calorimeter tower energies, subsequently used to provide the energies and directions of hadronic jets.

Jets are reconstructed offline from the energy deposits in the calorimeter towers, clustered using the anti-\kt algorithm~\cite{Cacciari:2008gp, Cacciari:2011ma} with a distance parameter of 0.4. In this process, the contribution from each calorimeter tower is assigned a momentum, the absolute value and the direction of which are given by the energy measured in the tower, and the coordinates of the tower. The raw jet energy is obtained from the sum of the tower energies, and the raw jet momentum by the vectorial sum of the tower momenta, which results in a nonzero jet mass. The raw jet energies are then corrected to establish a relative uniform response of the calorimeter in $\eta$ and a calibrated absolute response in transverse momentum \pt. 


 \section{Muon Reconstruction}\label{sec:CMSDetectorMuons}


Muons are measured in the pseudorapidity range $\abs{\eta} < 2.4$, with detection planes made using three technologies: drift tubes, cathode strip chambers, and resistive plate chambers. The single muon trigger efficiency exceeds 90\% over the full $\eta$ range, and the efficiency to reconstruct and identify muons is greater than 96\%. Matching muons to tracks measured in the silicon tracker results in a relative transverse momentum resolution, for muons with \pt up to 100\GeV, of 1\% in the barrel and 3\% in the endcaps. The \pt resolution in the barrel is better than 7\% for muons with \pt up to 1\TeV~\cite{Sirunyan:2018fpa}. 


A more detailed description of the CMS detector, together with a definition of the coordinate system used and the relevant kinematic variables, can be found in Ref.~\cite{Chatrchyan:2008zzk}.  The global event reconstruction (also called particle-flow event reconstruction~\cite{CMS-PRF-14-001}) is described in ~\ref{sec:PFReco}.




\chapter{Event Reconstruction}\label{chap:reco}

% \vspace{-5pt}
\section{Particle Flow Algorithm}\label{sec:ch3:PFlow}

Particle Flow is an algorithm which combines information from various CMS subdetectors and allows for a determination of the particle type based on observed properties.


\chapter{Jet Clustering and Grooming}\label{chap:jets}

\section{Introduction}
\label{sec:ch4:intro}
As noted previously, networked unmanned~~aerial~~vehicles~~(UAVs)~~have emerged as an important technology for public safety, commercial, and military applications including search and rescue, disaster relief, precision agriculture, environmental monitoring, and surveillance. Many of these applications require sophisticated mission planning algorithms to coordinate multiple drones to cover an area efficiently. Such scenarios are complicated by the existence of obstacles, such as buildings, requiring detailed planning for effective operation. Although a lot of work has been done on mission planning, optimal mission planning solutions depend heavily on the specific types of vehicles considered (e.g., ground robots, indoor drones, or outdoor drones), their kinematics, and the specific applications. Prior techniques have been optimized for shortest time to completion or control efficiency. However, a major challenge in the realization of such solutions is the limited energy on each drone. 




\chapter{Conclusion}\label{chap:conclusion}

\vspace{-5pt}
\section{Conclusion}\label{sec:concld}


The measurement matches the theoretical calculations well, i hope.


\center{The End.}
%%%%%%%%%%%%%%%%%%%%%%%%%%%%%%%%%%%%%%%%%%%%%%%%%%%%%%%%%%%%%%%
% Appendices
%
% Because of a quirk in LaTeX (see p. 48 of The LaTeX
% Companion, 2e), you cannot use \include along with
% \addtocontents if you want things to appear the proper
% sequence. Since the PSU Grad School requires
%%%%%%%%%%%%%%%%%%%%%%%%%%%%%%%%%%%%%%%%%%%%%%%%%%%%%%%%%%%%%%%
%\appendix
%

\chapter{Relativistic Kinematics}\label{chap:RelativisticKinematics}


\begin{figure}[htb]

\centering
\includegraphics[width=1.0\textwidth]{visuals/1to2splitting.png}
\caption{ A simple $ 1 \rightarrow 2  $ decay described by relativistic kinematics.}
\label{onetotwo}
\end{figure}

Consider the LO process depicted in ~\ref{onetotwo}, of a simple $ 1 \rightarrow 2  $ decay.


at LO there is a kinematic turn-off at $\pt R/2$ (where $R$ is the
distance parameter for the jet clustering), from the relativistic kinematics of a $1\rightarrow 2$ decay. 
However, for real jets the turn-off is closer to $\pt R/\sqrt{2}$ due to stochastic effects. 
To see this, consider a particle of energy $E$ and mass $m$ decaying to two massless
particles, each with an energy $E/2$ and separated by an angle $\theta$. 
The mass must satisfy $m^2 < \frac{E^2}{2}\left( 1 - \cos{\theta}\right)$. In the small
angle limit, this would be $m^2 < E^2\theta^2/4$, or $m < E\theta/2$.





According to relativistic kinematics the interaction can be described by the following equation:\newline

$p^{\mu} p_{\mu}  =  (p_1 + p_2)^{\mu} (p_1 + p_2)_{\mu}  $\newline

$p^{\mu} p_{\mu}  =  (p_1 + p_2)^{\mu} (p_1 + p_2)_{\mu}  $\newline

$m^2  = (E_1 + E_2)^2  - (\vec{p_1} + \vec{p_2} ) \dot (\vec{p_1} + \vec{p_2} )  $\newline


$m^2  \simeq 2 E_1 E_2 (1 - \cos \theta ) $\newline

If the energies of the splitting particles are equal then the equation simplifies since  $ E_1 = E_2 = \frac{E}{2} $ .


$m^2  < \frac{E^2}{2} ( 1 - \cos{\theta}  $\newline

$\frac{2m^2}{E^2}  < ( 1 - \cos{\theta}  $\newline

Using the small angle approximation this simplifies further.

$(1 - (1- \frac{\theta^2}{2}) )  \simeq  \frac{\theta^2}{2}  $\newline

Solving for mass, one finds :\newline

$m < E\theta/2$\newline

With more particle decays, the stochastic nature of the shower increases this to $m < E\theta/\sqrt{2}$.
Thus, a leading-order ($1\rightarrow 2$) decay will have a faster kinematic
turn-off than an all-orders ($1 \rightarrow$ many) decay.\newline

Solving for theta:\newline
 $\theta < \frac{2}{\gamma}$ where $\gamma$ is the lorentz factor $\gamma = \frac{1}{\sqrt{1-\frac{v^2}{c^2}}}  $.
%\chapter{Introduction}\label{chap:intro}

% \vspace{-5pt}
\section{Motivation}\label{sec:ch1:intro}

Within this dissertation, I provide a measurement of the differential jet cross section, as a function of the jet mass and transverse momentum, in events with a Z + Jet topology, with and without a jet grooming algorithm applied, using data collected by CMS experiment at LHC. The jet grooming used was the ``Soft Drop''
procedure 
%~\cite{softdrop}, wuth multiple values of the tunable parameters $\beta$ and $z_{cut}$. This represents the first, to my knowledge, measurement of it's kind with a light quark enriched jet sample at $\sqrt{s}$ = 13 TeV. 

Softdrop iteratively declusters a jet $j$ with distance parameter $R$ into two subjets, $j_1$ and $j_2$.
If the softdrop condition

\begin{equation}
  \frac{\min(p_{T1},p_{T2})}{p_{T1}+p_{T2}} > z_{cut} \cdot (\frac{\Delta R_{12}}{R})^\beta
\end{equation}

is met, then the procedure stops and $j$ is the final jet. Otherwise, the declustering continues - 
the higher $pt$ subjet is relabeled as $j$ and the lower $pt$ one is dropped.
By design, this condition fails for wide-angle soft radiation, which is therefore removed by the soft
drop procedure. The tunable parameters, $\beta$ and $z_{cut}$, control the degree of jet grooming:
$\beta$ tunes the algorithm's sensitivity to wide-angle radiation, while $z_{cut}$ sets the energy scale
of the grooming. In the case of $\beta \rightarrow \infty$, an ungroomed jet is returned. 
In the $\beta = 0$ case, the soft drop procedure is identical to the ``modified mass drop tagger'' (MMDT)
from Ref.
%~\cite{mmdt}. The soft drop algorithm removes soft and wide-angle radiation
from jets in a very theoretically controlled manner, making it suitable to separate
the ``hard'' and ``soft'' parts of the jet. Specifically, the soft drop
algorithm can remove non-global logarithms from correlations of
radiation within and between jets, which are extremely difficult to
compute theoretically
%~\cite{Dasgupta:2001sh,mmdt,softdrop,Dasgupta:2013via,Dasgupta:2015yua,Larkoski:2015zka}.

Comparing the production cross section for groomed and ungroomed jets separately allows us to
gain sensitivity to both the ``hard'' and ``soft'' jet physics. 
The groomed cross section can be directly compared to theoretical calculations of the jet mass
now and in the future, which is a very active area of theoretical research
at this time
%~\cite{Dasgupta:2012hg,Chien:2012ur,Jouttenus:2013hs,Almeida:2014uva,Liu:2014oog,Stewart:2014nna,Khelifa-Kerfa:2015mma,Frye:2016aiz,Kolodrubetz:2016dzb}. Furthermore, separating the hard and soft jet physics
allows a deeper understanding of the various effects involved in QCD
%radiation. In particular, Ref.~\cite{Frye:2016aiz} calculates the
groomed jet mass at next-to-next-to-leading order using soft colinear effective theory, matched to a
%parton shower at leading order using {\tt MCFM}~\cite{MCFM1,MCFM2}, and the authors of Ref.~\cite{mmdt} have
provided a next-to-leading logarithm calculation with traditional perturbative QCD, matched to a 
parton shower at leading order, also using {tt MCFM}.  We compare these theoretical predictions 
to our data in this paper for the first time in this channel at CMS. Both CMS and ATLAS have similar measurements in a dijet sample at %Ref.~\cite{cms_jetmassDijet, atlas_jetmass2}.

The analysis strategy is similar to that of %Ref.~\cite{cms_jetmassDijet}.
However, there are several differences. As in that paper, the cross section is also
unfolded in both jet mass and $pt$, however we also provide the measurement in jet $\rho$, dimensionless mass , and $pt$. While the previous measurement considered only one value for the soft drop parameter $\beta$, this analysis considers several.
We apply the soft drop algorithm to compare
directly to theoretical computations. Additionally, we not only measure the cross section as a function of mass, but also as a function of dimensionless mass, $\rho = 2log(m/(pt R))$, as is also done in the previously mentioned ATLAS measurement.  The dimensionless mass $\rho$ only weakly depends on $pt$, unlike mass, which is highly correlated. Additionally, the use of this variable aids in the separation of fixed order, perturbative and non-perturbative effects.
Finally, we also present the normalized differential cross section. We compute the cross sections normalized per $pt$ bin
(the ``normalized'' cross section) with respect to the jet $pt$ and jet mass 
by unfolding a binned two-dimensional distribution in $pt$ and mass
with widths $\Delta pt$ and $\Delta m$, respectively.

The normalized differential cross section in two forms :

\begin{equation}
\frac{1}{d\sigma/dpt}\frac{d^2\sigma}{dpt\,dm} = \frac{1}{N/\Delta pt} R(\frac{N_{ij}}{ \Delta pt \,\Delta m} )
\end{equation}


as well as :

\begin{equation}
\frac{1}{d\sigma/dpt}\frac{d^2\sigma}{dpt\,d\rho} = \frac{1}{N/\Delta pt} R(\frac{N_{ij}}{ \Delta pt \,\Delta \rho} )
\end{equation}


where $N$ is the total number of $Z+$jets events in our selection,
$N_{ij}$ is the number of such events in $pt$ bin $i$ and mass ($\rho$) bin $j$,
and $R(\alpha)$ is the unfolding procedure applied to the two-dimensional
distribution $\alpha$.

The 2 above normalized distributions are provided within for ungroomed  and groomed jets Anti-Kt Radious R$= 0.8$ jets. The groomed measurement is given in 9 configurations, one measurement is shown for jets groomed with every combination of 3 possible $\beta$ and $z_{cut}$ values (Where $\beta$ = 0 and $z_{cut}$ = 0 .1 is the current CMS default ):


$ \beta = [ 1,  0 , -1 ]  $

$ z_{cut}  = [ 0.15, 0.1, 0.05 ] $ 


These measurements currently represent humanity's highest energy measurement of a light quark enriched jet production cross section.


%%%%%%%%%%%%%%%%%%%%%%%%%%%%%%%%%%%%%%%%%%%%%%%%%%%%%%%%%%%%%%%
% ESM students need to include a Nontechnical Abstract as the %
% last appendix.                                              %
%%%%%%%%%%%%%%%%%%%%%%%%%%%%%%%%%%%%%%%%%%%%%%%%%%%%%%%%%%%%%%%
% This \include command should point to the file containing
% that abstract.
%\include{nontechnical-abstract}
%%%%%%%%%%%%%%%%%%%%%%%%%%%%%%%%%%%%%%%%%%%
} % End of the \allowdisplaybreak command %
%%%%%%%%%%%%%%%%%%%%%%%%%%%%%%%%%%%%%%%%%%%

%%%%%%%%%%%%%%%%
% BIBLIOGRAPHY %
%%%%%%%%%%%%%%%%
% You can use BibTeX or other bibliography facility for your
% bibliography. LaTeX's standard stuff is shown below. If you
% bibtex, then this section should look something like:
   \begin{singlespace}
   \bibliographystyle{unsrt}
   \phantomsection \addcontentsline{toc}{chapter}{Bibliography}
   \bibliography{ref}
   \end{singlespace}

%\begin{singlespace}
%\begin{thebibliography}{99}
%\addcontentsline{toc}{chapter}{Bibliography}
%\frenchspacing

%\bibitem{Wisdom87} J. Wisdom, ``Rotational Dynamics of Irregularly Shaped Natural Satellites,'' \emph{The Astronomical Journal}, Vol.~94, No.~5, 1987  pp. 1350--1360.

%\bibitem{G&H83} J. Guckenheimer and P. Holmes, \emph{Nonlinear Oscillations, Dynamical Systems, and Bifurcations of Vector Fields}, Springer-Verlag, New York, 1983.

%\end{thebibliography}
%\end{singlespace}

\backmatter

% Vita
% \vita{SupplementaryMaterial/Vita}

\end{document}

