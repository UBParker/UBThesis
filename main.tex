%%%%%%%%%%%%%%%%%%%%%%%%%%%%%%%%%%%%%%%%%%%%%%%%%%%%%%%%%%%%%%%
%
%     filename  = "Dissertation Index.tex",
%     version   = "Draft 1",
%     date      = "1/16/2013",
%     authors   = "Nicholas P. Nicoletti",
%     copyright = "Nicholas P. Nicoletti",
%     address   = "Department of Political Science,
%                  516 Park Hall,
%                  University at Buffalo,
%                  Buffalo, NY 14260,
%                  USA",
%     telephone = "(585) 752-5167",
%     email     = "npn@buffalo.edu",
%
%%%%%%%%%%%%%%%%%%%%%%%%%%%%%%%%%%%%%%%%%%%%%%%%%%%%%%%%%%%%%%%
%
% Change History:
%
% Draft Version 1.0 - No Changes.
%
%%%%%%%%%%%%%%%%%%%%%%%%%%%%%%%%%%%%%%%%%%%%%%%%%%%%%%%%%%%%%%%
%
% This is a template file to help get you started using the
% psuthesis.cls for theses and dissertations at Penn State
% University. You will, of course, need to put the
% psuthesis.cls file someplace that LaTeX will find it.
%
% We have set up a directory structure that we find to be clean
% and convenient. You can readjust it to suit your tastes. In
% fact, the structure used by our students is even a little
% more involved and commands are defined to point to the
% various directories.
%
% This document has been set up to be typeset using pdflatex.
% About the only thing you will need to change if typesetting
% using latex is the \DeclareGraphicsExtensions command.
%
% The psuthesis document class uses the same options as the
% book class. In addition, it requires that you have the
% ifthen, calc, setspace, and tocloft packages.
%
% The first additional option specifies the degree type. You
% can choose from:
%     Ph.D. using class option <phd>
%     M.S. using class option <ms>
%     M.Eng. using class option <meng>
%     M.A. using class option <ma>
%     B.S. using class option <bs>
%     B.A. using class option <ba>
%     Honors Baccalaureate using the option <honors>
%
% If you specify either ba or bs in addition to honors, it will
% just use the honors option and ignore the ba or bs option.
%
% The second additional option <inlinechaptertoc> determines
% the formatting of the Chapter entries in the Table of
% Contents. The default sets them as two-line entries (try it).
% If you want them as one-line entries, issue the
% inlinechaptertoc option.
%
% The class option ``honors'' should be used for theses
% submitted to the Schreyer Honors College. This option
% changes the formatting on the Title page so that the
% signatures appear on the Title page. Be sure and comment
% out the command \psusigpage when using this option since it
% is not needed and it messes up the vertical spacing on the
% Title page.
%
% The class option ``honorsdepthead'' adds the signature of the
% department head on the Title page for those baccalaureate
% theses that require this.
%
% The class option ``secondthesissupervisor'' should be used
% for baccalaureate honors degrees if you have a second
% Thesis Supervisor.
%
% The vita is only included with the phd option and it is
% placed at the end of the thesis. The permissions page is only
% included with the ms, meng, and ma options.
%%%%%%%%%%%%%%%%%%%%%%%%%%%%%%%%%%%%%%%%%%%%%%%%%%%%%%%%%%%%%%%
% Only one of the following lines should be used at a time.
\documentclass[phd,12pt]{psuthesis}
%\documentclass[draft,phd,inlinechaptertoc]{psuthesis}
%\documentclass[draft,ms]{psuthesis}
%\documentclass[draft,honorsdepthead,honors]{psuthesis}
%\documentclass[draft,honors]{psuthesis}
%\documentclass[draft,secondthesissupervisor,honors]{psuthesis}
%\documentclass[draft,bs]{psuthesis}%


%%%%%%%%%%%%%%%%%%%%%%%%%%%%
% Packages we like to use. %
%%%%%%%%%%%%%%%%%%%%%%%%%%%%
%\usepackage{ptdr-definitions}
% Just stole these from utils/general/ptdr-definitions.sty
\newcommand{\PT}{\ensuremath{p_{\mathrm{T}}}\xspace}
\newcommand{\pt}{\ensuremath{p_{\mathrm{T}}}\xspace}
\newcommand{\MADGRAPH}{\textsc{MadGraph}\xspace}
\newcommand{\PYTHIA}{{\textsc{pythia}}\xspace}
\newcommand{\kt}{\ensuremath{k_{\mathrm{T}}}\xspace}
\newcommand{\FASTJET}{{\textsc{FastJet}}\xspace}
\newcommand{\GeV}{\ensuremath{\,\text{Ge\hspace{-.08em}V}}\xspace}
\newcommand{\GeVns}{\ensuremath{\text{Ge\hspace{-.08em}V}}\xspace} % no leading thinspace
\newcommand{\gev}{\GeV}
\newcommand{\TeV}{\ensuremath{\,\text{Te\hspace{-.08em}V}}\xspace}
\newcommand{\MET}{\ETm}
\newcommand{\ttbar}{\ensuremath{\mathrm{t} \overline{\mathrm{t}}}\xspace} % t-tbar
\newcommand{\cPqt}{\PQt} % t for t quark
\newcommand{\cPaqt}{\PAQt} % t for t anti-quark
\newcommand{\ETm}{\ensuremath{E_{\mathrm{T}}^{\text{miss}}}\xspace}

\usepackage{amsmath}
\usepackage{amssymb}
\usepackage{amsthm}
\usepackage{exscale}
\usepackage[mathscr]{eucal}
\usepackage{bm}
\usepackage{eqlist} % Makes for a nice list of symbols.
\usepackage[final]{graphicx}
\usepackage[dvipsnames]{color}
\DeclareGraphicsExtensions{.pdf, .jpg, .png}
%\usepackage{natbib}
%\bibliographystyle{unsrtnat}
%\setcitestyle{numbers, square}


\usepackage[utf8]{inputenc}
\usepackage[english]{babel} 
\usepackage[square,numbers]{natbib}
\bibliographystyle{unsrt}
 

%\addbibresource{ref} % The filename of the bibliography


\usepackage{har2nat}
\usepackage{verbatim}
\usepackage{url}
\usepackage{longtable}
\usepackage{mathpazo}
\usepackage{pstricks}
\usepackage{sgamevar}
\usepackage{egameps}
%\usepackage{natbib}
%\addbibresource{athesis}

%%%%%%%%%%%%%%%%%%%%%%%%
% Setting for fncychap %
%%%%%%%%%%%%%%%%%%%%%%%%
% Comment out or remove the next two lines and you will get
% the standard LaTeX chapter titles. We like these A LOT
% better.
\usepackage[Lenny]{fncychap}
\ChTitleVar{\Huge\sffamily\bfseries}


%%%%%%%%%%%%%%%%%%%%%%%%%%%%%%%
% Use of the hyperref package %
%%%%%%%%%%%%%%%%%%%%%%%%%%%%%%%
%
% This is optional and is included only for those students
% who want to use it.
%
% To use the hyperref package, uncomment the following line:
%\usepackage[colorlinks=true,urlcolor=purple,citecolor=green,linkcolor=blue]{hyperref}
%
% Note that you should also uncomment the following line:
%\renewcommand{\theHchapter}{\thepart.\thechapter}
%
% to work around some problem hyperref has with the fact
% the psuthesis class has unnumbered pages after which page
% counters are reset.


%%%%%%%%%%%%%%%%%%%%%%%%%%%%%%%%%%%%
% SPECIAL SYMBOLS AND NEW COMMANDS %
%%%%%%%%%%%%%%%%%%%%%%%%%%%%%%%%%%%%
\input{SupplementaryMaterial/UserDefinedCommands}


%%%%%%%%%%%%%%%%%%%%%%%%%%%%%%%%%%%%%%%%%
% Renewed Float Parameters              %
% (Makes floats fit better on the page) %
%%%%%%%%%%%%%%%%%%%%%%%%%%%%%%%%%%%%%%%%%
\renewcommand{\floatpagefraction}{0.85}
\renewcommand{\topfraction}      {0.85}
\renewcommand{\bottomfraction}   {0.85}
\renewcommand{\textfraction}     {0.15}

% ----------------------------------------------------------- %

%%%%%%%%%%%%%%%%
% FRONT-MATTER %
%%%%%%%%%%%%%%%%
% Title
\title{Differential Jet Production Cross Section Measurement in Z + Jet events from Proton - Proton Collisions at $\sqrt{s}$ = 13 TeV using the CMS detector at LHC}

% Author and Date of Degree Conferral or Defense
\author{Ashley Marie Parker}
% the degree will be conferred on this date
\degreedate{January 2020}
% year of your copyright. I have removed this from the cover page because UB's guidelines do not include it.
%\copyrightyear{2019}

% This is the document type. For example, this could also be:
%     Comprehensive Document
%     Thesis Proposal
\documenttype{Disseration}
%The department where you will be submitting the document%
\dept{Department of Physics}
% This will generally be The Graduate School, though you can
% put anything in here to suit your needs. This has also been removes from the cover page via the psuthesis.cls document because UB guidelines do not allow for it.
\submittedto{The Graduate School}


%%%%%%%%%%%%%%%%%%
% Signatory Page %
%%%%%%%%%%%%%%%%%%
% You can have up to 7 committee members, i.e., one advisor
% and up to 6 readers.
%
% Begin by specifying the number of readers.
\numberofreaders{3}


% Input reader information below. The optional argument, which
% comes first, goes on the second line before the name.
\advisor[Thesis Advisor, Chair of Committee]
        {Salvatore Rappoccio}
        {Professor of Physics}

\readerone[Committee Member]
          {Ia Iashvili}
          {Professor of  Physics}

\readertwo[Committee Member]
          {Ciaran Williams}
          {Professor of  Physics}




% Makes use of LaTeX's include facility. Add as many chapters
% and appendices as you like.
\includeonly{
Chapter-0/chap0,% introduction ****CALLED Ch1 all off by 1 bc I started with 0 here****
Chapter-1/chap1,% Theoretical Framework
Chapter-2/chap2,%  jets in pp collisions
Chapter-3/chap3,% CMS experiment 
Chapter-4/chap4,% event generation and reconstruction
Chapter-5/chap5,% Z + jets AN 
Chapter-6/chap6% W tagging AN
%Appendix-A/main,%
%Appendix-B/main%
}

%%%%%%%%%%%%%%%%%
% THE BEGINNING %
%%%%%%%%%%%%%%%%%
\begin{document}

%%%%%%%%%%%%%%%%%%%%%%%%
% Preliminary Material %
%%%%%%%%%%%%%%%%%%%%%%%%
% This command is needed to properly set up the frontmatter.
\frontmatter

%%%%%%%%%%%%%%%%%%%%%%%%%%%%%%%%%%%%%%%%%%%%%%%%%%%%%%%%%%%%%%
% IMPORTANT
%
% The following commands allow you to include all the
% frontmatter in your thesis. If you don't need one or more of
% these items, you can comment it out. Most of these items are
% actually required by the Grad School -- see the Thesis Guide
% for details regarding what is and what is not required for
% your particular degree.
%%%%%%%%%%%%%%%%%%%%%%%%%%%%%%%%%%%%%%%%%%%%%%%%%%%%%%%%%%%%%%
% !!! DO NOT CHANGE THE SEQUENCE OF THESE ITEMS !!!
%%%%%%%%%%%%%%%%%%%%%%%%%%%%%%%%%%%%%%%%%%%%%%%%%%%%%%%%%%%%%%

% Generates the signature page. This is not bound with your
% thesis.
%\psusigpage

\pagestyle{plain}
\pagenumbering{roman}

% Generates the title page based on info you have provided
% above.
\psutitlepage

%Generates Copyright Page
%\copyrightpage{SupplementaryMaterial/Copyright}

\newpage
% Generates the committee page -- this is bound with your
% thesis. If this is an baccalaureate honors thesis, then
% comment out this line.
% \psucommitteepage

% Generates the Epigraph/Dedication. The first argument should
% point to the file containing your Epigraph/Dedication and
% the second argument should be the title of this page.
%\thesisdedication{SupplementaryMaterial/Dedication}{Dedication}

% Generates the Acknowledgments. The argument should point to
% the file containing your Acknowledgments.
%\thesisacknowledgments{SupplementaryMaterial/Acknowledgments}

% Generates the Table of Contents
\thesistableofcontents

% Generates the List of Tables
\thesislistoftables

% Generates the List of Figures
\thesislistoffigures

% Generates the List of Symbols. The argument should point to
% the file containing your List of Symbols.
% \thesislistofsymbols{SupplementaryMaterial/ListOfSymbols}

% Generates the abstract. The argument should point to the
% file containing your abstract.
\thesisabstract{SupplementaryMaterial/Abstract.tex}


\pagenumbering{arabic}

%%%%%%%%%%%%%%%%%%%%%%%%%%%%%%%%%%%%%%%%%%%%%%%%%%%%%%
% This command is needed to get the main part of the %
% document going.                                    %
%%%%%%%%%%%%%%%%%%%%%%%%%%%%%%%%%%%%%%%%%%%%%%%%%%%%%%
\thesismainmatter

%%%%%%%%%%%%%%%%%%%%%%%%%%%%%%%%%%%%%%%%%%%%%%%%%%
% This is an AMS-LaTeX command to allow breaking %
% of displayed equations across pages. Note the  %
% closing the "}" just before the bibliography.  %
%%%%%%%%%%%%%%%%%%%%%%%%%%%%%%%%%%%%%%%%%%%%%%%%%%
\allowdisplaybreaks{
%
%%%%%%%%%%%%%%%%%%%%%%
% THE ACTUAL CONTENT %
%%%%%%%%%%%%%%%%%%%%%%
% Chapters
%\label{ch1:intro}

%\vspace{-5pt}

%%\section{Motivation}\label{secMot:ch1}

%Within this dissertation, I provide a measurement of the differential jet cross section, as a function of the jet mass and transverse momentum, in events with a Z + Jet topology using data collected by CMS experiment at LHC. 



















\chapter{Introduction}\label{sec:intro}


%a measurement of the double differential jet production cross section as a function of the jet mass and transverse momentum, in events with a Z + Jet topology, using 136 $fb^{-1}$ of data acquired from the Compact Muon Solenoid (CMS) experiment at the Large Hadron Collider (LHC) during Run 2 of data taking. This measurement improves upon the previous measurement by CMS experiment at 7 TeV center-of-mass energy ~\cite{Chatrchyan:2013vbb} and the analogous dijet measurement at 13 TeV ~\cite{Sirunyan:2018xdh} . There is a similar measurement by ATLAS experiment in the dijet channel ~\cite{Aaboud:2017qwh}. This is the first measurement in the Z + Jets channel to be presented at 13 TeV. %A leading source of systematic uncertainty in the measurement is the jet mass scale which can be improved upon using studies such as the one in chapter \ref{chap:Wtag}

%Studying Z + jet events will yield a light quark enriched jet sample, which has not yet been studied at $\sqrt{s}$ = 13 TeV. Comparing groomed and ungroomed jets will allow us the better understand the jet mass. The grooming was acheived using jet grooming techniques discussed in ~\ref{sec:jetgroom}, leading to jets with less soft and collinear radiation with respect to the ungroomed counterpart in varying degrees depending on the grooming algorithm and choice of algorithm specific parameters. 

This thesis presents a measurement of the differential production cross section
of $Z$+jets events as a function of the jet mass and transverse
momentum ($\pt$). The cross section is presented for events before
and after the jets are groomed with the ``soft drop''
procedure~\cite{softdrop}, using multiple values of the tunable parameters $\beta$ and $z_{cut}$. Each unique combination of the tunable parameters $\beta$ and $z_{cut}$ leads to a jet with less soft and collinear radiation with respect to the ungroomed counterpart in varying degrees depending on the choice. The soft drop grooming algorithm is described in more detail in Section~\ref{sec:jetgroom}.


Comparing the production cross section for groomed and ungroomed jets separately allows us to, in the language of Soft-Collinear Effective Theory (SCET) described in Section~\ref{sec:SCET},
gain sensitivity to both the ``hard'' and ``soft'' jet physics. 
The groomed cross section can be directly compared to theoretical calculations of the jet mass
now and in the future, which is a very active area of theoretical research
at this time~\cite{Dasgupta:2012hg,Chien:2012ur,Jouttenus:2013hs,Almeida:2014uva,Liu:2014oog,Stewart:2014nna,Khelifa-Kerfa:2015mma,Frye:2016aiz,Kolodrubetz:2016dzb}. Furthermore, separating the hard and soft jet physics
allows a deeper understanding of the various effects involved in QCD
radiation. In particular, Ref.~\cite{Frye:2016aiz} calculates the
groomed jet mass at next-to-next-to-leading order using soft colinear effective theory, matched to a
parton shower at leading order using {\tt MCFM}~\cite{MCFM1,MCFM2}, and the authors of Ref.~\cite{mmdt}  are preparing a next-to-leading logarithm calculation with traditional perturbative QCD, matched to a 
parton shower at leading order, also using {\tt MCFM}.  In the near future we will compare these theoretical predictions to our data measurement. Both CMS and ATLAS have similar measurements in a dijet sample at Ref.~\cite{cms_jetmassDijet, atlas_jetmass2}.

The analysis strategy is similar to that of Ref.~\cite{cms_jetmassDijet}.
However, there are several differences. As in that paper, the cross section is now
unfolded in both jet mass and $\pt$. However, while the previous measurement
considered only one value for the soft drop parameter $\beta$, this analysis considers several.
We apply the soft drop algorithm to compare directly to theoretical computations. Additionally, we not only measure the cross section as a function of mass, but also as a function of dimensionless mass, $\rho = 2log(m/(pt R))$, as is also done in the previously mentioned ATLAS measurement.  The dimensionless mass $\rho$ only weakly depends on $\pt$, unlike mass, which is highly correlated. Additionally, the use of this variable aids in the separation of fixed order, perturbative and non-perturbative effects.
We present the normalized double differential jet production cross section with respect to jet mass and transverse momentum ($\pt$ ) as well as with respect to jet dimensionless mass and $\pt$ . We compute the cross sections normalized per $\pt$ bin
(the ``normalized'' cross section) with respect to the jet $\pt$ and jet mass 
by unfolding a binned two-dimensional distribution in $\pt$ and mass
with widths $\Delta pt$ and $\Delta m$, respectively.

The normalized differential cross section

\begin{equation}
\frac{1}{d\sigma/dpt}\frac{d^2\sigma}{dpt\,dm} = \frac{1}{N/\Delta pt} R(\frac{N_{ij}}{ \Delta pt \,\Delta m} )
\end{equation}

where $N$ is the total number of $Z+$jets events in our selection,
$N_{ij}$ is the number of such events in $pt$ bin $i$ and mass bin $j$,
and $R(\alpha)$ is the unfolding procedure applied to the two-dimensional
distribution $\alpha$.

\chapter{Theoretical Framework}%\label{ch1:intro}

%\vspace{-5pt}

%%\section{Motivation}\label{secMot:ch1}

%Within this dissertation, I provide a measurement of the differential jet cross section, as a function of the jet mass and transverse momentum, in events with a Z + Jet topology using data collected by CMS experiment at LHC. 


















%%%%%%%%%%%%%%%%%%%%%%%%%%%%%%%%%%%%%%%%%%%%%%%%%%%%%%
\section{Introduction To The Standard Model}\label{secSM}



The Standard Model (SM) of particle physics is a quantum field theory (QFT) description of the strong, weak and electromagnetic forces of nature. The known particles of the SM are: 1 scalar Higgs boson, 4 gauge bosons, 6 types of quarks and 6 types of leptons. 

The quarks and leptons are fermions that constitute matter and so obey Fermi-Dirac statistics due to their half-integer spin. In contrast, the bosons have integer spin and obey Bose-Einstein statistics. Gauge bosons mediate the 3 fundamental forces and the Higg's boson is responsible for the electro-weak symmetry breaking which gives mass to the other particles~\cite{Griffiths:111880}. 

% rivello: Error below on \cite{fig:SM}
The fermions are arranged into 3 generations, arranged in columns from left to right on ~\ref{fig:SM}

\begin{figure}[htb]
\centering
\includegraphics[width=.60\textwidth]{smdiagram.pdf}
\caption{Fundamental particles of the Standard Model~\cite{modellinginvisible}.}
\label{fig:SM}
\end{figure}




The SM is humanity's most rigorous theory of our universe, providing predictions of observables that have since been measured, in the case of Quantum Electrodynamics, QED, to the highest precision of any scientific theory. Despite the impressive predictions, the gravitational force and more subtle phenomena, such as flavor oscillation of neutrinos ~\cite{Ashie:2005ik}, indicate the existence of physics beyond the standard model, BSM.

Various attempts have been made to unify the fundamental forces under one theory, thus far the electromagnetic and weak interactions have been united by electro-weak theory. 

The Standard Electroweak Model can be described $SU(2) x U(1)$ mathematically.
% rivello: remember to fix paragraph below.
The  $SU(2) x U(1)$ gauge group is a unification of the special unitary symmetry group $SU(2)$ describing 3 mixed massive vector bosons, ($W_{-}$ $W_{+}$ $Z_0$), as carriers of the weak nuclear force, and the unitary gauge group $U(1)$ , describing the massless chargeless photon, of the electromagnetic interaction.

The standard model of the strong interaction is known as Quantum Chromodynamics, QCD, a non-Abelian gauge theory described by the special unitary group $(SU(3)_c)$, where the flavours of quark are the physical manifestation of the symmetry group. This force is mediated by the 8 massless gluons that carry color charge, making QCD more complicated mathematically than QED.

The SM also contains a Higgs boson, an excitation of a scalar Higg's field, which gives rise to spontaneous symmetry breaking of the electroweak theory, providing the particles with mass, but that won't be discussed herein. 

The quarks and leptons are arranged in generations according to their relative masses, as shown in Figure \ref{fig:SM}. The table also shows the spins of the particles, the leptons and quarks have half-integer spin, fermions, that obey the fermi exclusion principle. Conversely, the bosons have half integer spin and therefore obey bose-einstein statistics. Through the SM we interpret the observed hadronic particles, mesons ( baryons ), as 2 quark (3 quark) bound states. The existence of spin $\frac{3}{2}$ baryons, which are symmetric bound states in space, spin and flavour, and the need to obey Fermi-Dirac statistics, by maintaining total assymmetry of the wave function, implies there is another degree of freedom, called color, so that each quark is either red, green or blue. Granted only color singlet states exist. Furthermore there exists a property of asymptotic freedom where the QCD coupling between quarks and gluons increases as they asymptotically approach one another. There exist a wealth of experimental data to support the concept of asymptotic freedom. Asymptotic freedom is a useful property as it allows for perturbative calculations of QCD observables, such as the jet mass, this is discussed in ~\ref{sec:jetmass}. The running of the QCD (strong) coupling constant as measured by CMS experiment can be seen in ~\ref{fig:alphas}


\begin{figure}[htb]
\centering
\includegraphics[width=.70\textwidth]{visuals/strong-coupling-cms2.png}
\caption{The running of the strong coupling constant as compiled by CMS including measurements from CMS and HERA among others~\cite{CMS:2014mna}.}
\label{fig:alphas}
\end{figure}


%image CMS:2014mna
% visuals/strong-coupling-cms






%another DY thesis  http://inspirehep.net/record/1345977/files/DoolingSamantha_Dissertation.pdf


% Symmetris imply conserved quantities, Neuther's Theorem


Nuclei in ordinary matter are composed solely of $1^{st}$ generation particles, up and down quarks, bound by gluons. Neutral atoms contain an equal number of protons (composed of 2 up quarks and a down quark) and electrons, $1^{st}$ generation leptons. The main distinction between leptons and quarks, both fermions (particles of $\frac{1}{2}$ integer spin), being that leptons do not experience the color interaction $(SU(3)_c)$ like their quark friends. In each generation there is a quark with charge $Q = + \frac{2}{3}$ (up, charm, top) and another of charge $Q = - \frac{1}{3}$ (down, strange, bottom).





\subsection{Quantum Chromodynamics}\label{secQCD}


QCD is a quantum field theory that describes the color force, experienced by quarks and mediated by gluons. The quarks each posess a color charge ; red, green or blue (anti-red, anti-green or anti-blue). In contrast the gluons are "bicolored", each carrying one unit of negative and one of positive charge.~\cite{Griffiths:111880}. As previously metioned, QCD a non-Abelian gauge theory described by the special unitary group $(SU(3)_c$ (color charge) and has a Lagrangian that can be written as follows:\newline


\begin{equation}
\mathcal{L}=-\frac{1}{4} F_{\mu \nu}^{A} F_{A}^{\mu \nu}+\sum_{\text {flavours }} \overline{\psi}_{a}\left(i \gamma_{\mu} D^{\mu}-m\right)_{a b} \psi_{b}
\end{equation}


The sum is over each of the generators of the $(SU(3)_c$ gauge group, and the quark field fermion multiplets, $\psi$, belong to its irreducible representation.~\cite{Crewther:1995wq}

The gluon fields $A_{\nu}^a$ of spin 1 have a field strength tensor, $F_{\mu \nu}^{A}$ , given below:\newline 

\begin{equation}
F_{\mu \nu}^{A}=\partial_{\mu} A_{\nu}^{A}-\partial_{\nu} A_{\mu}^{A}+g_{s} f^{A B C} A_{\mu}^{B} A_{\nu}^{C}
\end{equation}


The structure functions are denoted as $f^{A B C}$ and their indicies run over all of the gluon color degrees of freedom. It is noteable to mention that the third term in the above equation is what gives rise to asymptotic freedom through the gluon quartic and triple self-interactions it induces ~\cite{Crewther:1995wq}.

The covariant derivative is defined as :\newline

\begin{equation}
\left(D_{\mu}\right)_{a b}=\partial_{\mu} \delta_{a b}-i g_{s} A_{\mu}^{A} t_{a b}^{A}
\end{equation}

Where the $t_{a}$ are the matrices of the fundamental representation of $(SU(3)$.

Lastly, for completeness, I mention the final gauge invariant term of the QCD Lagrangian below, where $\theta_{QCD}$ is a free parameter of QCD known as the vacuum angle parameter.

\begin{equation}
\mathcal{L}_{\theta}=\theta_{\mathrm{QCD}} \frac{\alpha_{s}^{2}}{64 \pi^{2}} \epsilon^{\mu \nu \rho \sigma} F_{\mu \nu}^{a} F_{\rho \sigma}^{a}
\end{equation}


QCD is discussed from a phenomenological perspective in ~\ref{sec:quarkandgluonjets}. 








%%%%%%%%%%%% NEW  CHAPTER %%%%%%%%%%%%%%%%%%%%







%%%%%%%%%%%%%%%%%%%%%%%%%%%%%%%%%%%%%%%%%%%%%%%%%%%%%%


\section{Jet and Soft Functions in Soft-Collinear Effective Theory}
%https://arxiv.org/pdf/1410.1892.pdf

Below the factorization structure of the double differential jet production cross section is displayed in the context of Soft-Collinear Effective Theory, SCET, following the framework for inclusive jet production $pp \rightarrow jet + X$ developed in for jets of  %https://arxiv.org/abs/1801.00790

\begin{equation}
\frac{d \sigma}{ d p_{T} d m}=\sum_{a b c} f_{a}\left(x_{a}, \mu\right) \otimes f_{b}\left(x_{b}, \mu\right) \otimes H_{a b}^{c}\left(x_{a}, x_{b}, p_{T} / z, \mu\right) \otimes \mathcal{G}_{c}\left(z, p_{T} R, m, \mu, z_{\mathrm{cut}}, \beta\right)
\end{equation}
% equation from here https://arxiv.org/pdf/1811.06983.pdf



Image of factorization in this context

%SCET-factorization
~\cite{Becher:2014oda}
%https://arxiv.org/abs/1410.1892
\begin{figure}[htb]
\centering
\includegraphics[width=1.0\textwidth]{visuals/SCET-factorization.png}
\caption{Fatorization of the energy scales in a hard scatter interaction according to SCET ~\cite{Becher:2014oda}.}
\label{fig:scet}
\end{figure}
discuss groomed jets in this context

comparing groomed  (Jet) to ungroomed (Jet+Soft)

\section{Jets Initited by Quarks and Gluons }\label{jetgroom:ch1}



earlier discussed $C_F = \frac{4}{3}$ and $C_A=3$ 

CITE A Theory of Quark vs. Gluon Discrimination
% https://arxiv.org/pdf/1906.01639.pdf

Dijets make quark gluon admixture %cite SMP-16-10
Z+Jets make mostly light quark jets, studied here and in 7 TeV analysis (1 D unfolding there and no soft drop)

% http://cms-results.web.cern.ch/cms-results/public-results/publications/SMP-16-010/index.html




% This similarity allows us to interpret the variations in quark enriched and gluon enriched jet samples in terms of the fundamental $C_F$ and adjoint $C_A$ casimirs, in $SU(3)$ ,   $C_F = \frac{4}{3}$ and $C_A=3$. 


%Comparing the probability of a quark to emit a gluon and that of a gluon to to emit a gluon we can see the ratio will give simply $\frac{C_A}{C_F} =\frac{9}{4} $. This has strong experimental implications since it implies gluon jets will on average be composed of about twice as many constituent particles as quark jets.



% ATLAS thesis http://inspirehep.net/record/1672323/files/2016_Mantifel_PhD_Atlas_Z.pdf








\chapter{CMS Experiment at LHC}\label{chap:CMS}

\vspace{-3pt}
\section{The Large Hadron Collider}\label{sec:ch2:lhc}

The Large Hadron Collider, LHC, is the largest machine created my mankind.





\vspace{-3pt}
\section{The CMS Detector}\label{sec:CMSDetector}



The Compact Muon Solenoid (CMS) detector was used to collect the data presented in this thesis, it is one of two large general purpose detectors at the LHC. CMS experiment has recorded 162 $fb^{-1}$ integrated luminosity in the dataset presented in this thesis, collected during Run 2 of LHC.

The Compact Muon Solenoid, CMS, is one of 4 detectors that measure collisions of protons and lead ions produced by the Large Hadron Collider, LHC, at CERN. CMS is the smaller of the 2 large general-purpose detectors, the other being ATLAS. The most notable feature of the detector is it's powerful 3.8 Tesla solenoid magnet, the largest superconducting magnet ever built, as of the year 2011.


The central feature of the CMS apparatus is a superconducting solenoid of 6 m internal diameter, providing a magnetic field of 3.8 T. Within the solenoid volume are a silicon pixel and strip tracker, a lead tungstate crystal electromagnetic calorimeter, and a brass and scintillator hadron calorimeter, each composed of a barrel and two endcap sections. Forward calorimeters, made of steel and quartz-fibres, extend the pseudorapidity coverage provided by the barrel and endcap detectors. Muons are detected in gas-ionization chambers embedded in the steel flux-return yoke outside the solenoid. 

%https://twiki.cern.ch/twiki/bin/viewauth/CMS/Internal/PubDetector



% riju https://adasgupt.web.cern.ch/adasgupt/locked/thesis.pdf



%%% Event generation and reconstruction chapter

\chapter{Event Generation and Reconstruction}\label{chap:MCGenReco}


The events used for this measurement were reconstructed from data acquired by the CMS detector in the case of the data and generated using Monte Carlo generators PYTHIA and HERWIG in the case of the generated data.

\section{Brief Introduction to Monte Carlo Event Generators}\label{secMCGen}

Monte Carlo (MC) event generators are tools used by both experimental and theoretical physicists to simulate different physical processes in order to make predictions and prepare future experiments. The main tasks of such generators are to calculate matrix elements of the relevant hard processes but they must also describe parton showering, hadronization and underlying event. MC can be utilized to extrapolate data measurements beyond the acceptance of the detector or in the case of this thesis it is used in the unfolding process to correct the data for detector efficiency and resolution.


MC generators provide an ensemble of generated events that simulate the physics process in question. Each individual generator implements a slightly different scheme in order to approximate the necessary calculations for the factorization and renormalization scales relevant to a process. These variations mean that the choice of MC generator will have a slight effect on the generated distributions. For this reason, in the analysis described herein we compared results from 2 different generators: PYTHIA and HERWIG.



All event generators break the calculations into separate parts as depicted by the different colored objects in Figure ~\ref{fig:MCeventpretty}  


\begin{figure}[htb]
\centering
\includegraphics[width=1.0\textwidth]{visuals/MCeventpretty.png}
\caption{A pictorial representation of the way the MC generators simulate hadron-hadron collisions~\cite{Hoche:2014rga}.}
\label{fig:MCeventpretty}
\end{figure}





PYTHIA ~\cite{Sjostrand:2006za} is a very commonly used general purpose event generator which uses the parton shower approach for higher order corrections to the hard scattering matrix element.

HERWIG  ~\cite{herwigpp} is another commonly used event generator, incredibly similar to PYTHIA, differing mainly in hadronization and parton showering behaviors.


\section{Event Reconstruction with Particle Flow}\label{sec:PFReco}


The data from proton-proton collisions at LHC are detected by the CMS experiment and then the Particle Flow (PF) event reconstruction ~\cite{Sirunyan:2017ulk} is applied to these raw detector outputs in order to construct "particle flow objects" that contain information from multiple CMS detector subsystems and constitute a global event description. More details about the CMS apparatus can be found in ~\ref{chap:CMS}. The particle flow objects are given defined object categories based on which subsystems they are measured in. This requires the use of a "link algorithm" which combines subdetector information together into a single object ~\cite{Sirunyan:2017ulk}. For example, one can see that the solid, light blue, line in figure ~\ref{fig:cmsPF} corresponds to a muon PF object as it left hits in the tracker then traversed all of the calorimeters only to deposit its energy in the muon chambers.


\begin{figure}[htb]
\centering
\includegraphics[width=1.0\textwidth]{Chapter-1/cmsPflow.png}
\caption{A pictorial representation of the way the Particle Flow algorithm determines which objects correspond to which particles based on an optimal combination of sub-detector information ~\cite{Sirunyan:2017ulk}.}
\label{fig:cmsPF}
\end{figure}


 PF aims to reconstruct and identify each individual particle in an event, with an optimized combination of all subdetector information. In doing so PF characterizes the particles into 5 types: photon, electron, muon, charged hadron, neutral hadron. The type determines which sub-detector information will be combined to determine the energy and direction of that particle. Photons (\eg coming from electron bremsstrahlung) are identified as ECAL energy clusters that are not linked to the extrapolation of any charged particle trajectory to the ECAL. Electrons (\eg coming from photon conversions in the tracker material or from semileptonic decays of hadrons) are identified as a primary charged particle track and potentially many ECAL energy clusters corresponding to this track extrapolation to the ECAL and to any bremsstrahlung photons emitted along the way within the tracker. Muons (\eg from  hadron semileptonic decays) are identified as tracks in the central tracker consistent with either a track or several hits in the muon system, and associated with calorimeter deposits compatible with the muon hypothesis. Charged hadrons are identified as charged particle tracks neither identified as electrons, nor as muons. Finally, neutral hadrons are identified as HCAL energy clusters not linked to any charged hadron trajectory, or as a combined ECAL and HCAL energy excess with respect to the expected charged hadron energy deposit.

The ECAL is used to obtain the energy of photons. The energy of electrons is more complex, determined from a combination of the track momentum at the main interaction vertex, the corresponding ECAL cluster energy, and the energy sum of all bremsstrahlung photons originating from the track. The energy of muons is obtained from the corresponding track momentum. The energy of charged hadrons is determined from a combination of the track momentum and the corresponding ECAL and HCAL energies, corrected for zero-suppression effects and for the response function of the calorimeters to hadronic showers. Finally, the energy of neutral hadrons is obtained from the corresponding corrected ECAL and HCAL energies.

After particle flow is used to determine the particle's type, the particle flow objects which are not categorized as muons, electrons and isolated photons are then described as hadrons and are clustered into "jets". An example PF jet with 5 constituent PF objects is depicted in ~\ref{fig:cmsPFjet}.

\begin{figure}[htb]
\centering
\includegraphics[width=1.0\textwidth]{Chapter-1/cmsPflowjet.png}
\caption{A pictorial representation of the way the Particle Flow objects are clustered into jets ~\cite{Sirunyan:2017ulk}. The top image shows a jet composed of 5 PF candidates in the x,y plane and the concentric circles describe the surfaces of the ECAL and HCAL detectors respectively. The lower left image is the PF candidates as measured by the ECAL and the image on the lower right shows the same for the HCAL surface. The dotted lines represent generated particles and the solid boxes represent energy deposits in the detector. }
\label{fig:cmsPFjet}
\end{figure}

For each event, hadronic jets are clustered from these reconstructed particles using the infrared and collinear safe anti-\kt algorithm~\cite{Cacciari:2008gp, Cacciari:2011ma} with a distance parameter of 0.4. Jet momentum is determined as the vectorial sum of all particle momenta in the jet, and is found from simulation to be, on average, within 5 to 10\% of the true momentum over the whole \pt spectrum and detector acceptance. Additional proton-proton interactions within the same or nearby bunch crossings (pileup) can contribute additional tracks and calorimetric energy depositions to the jet momentum. 

The pileup per particle identification (PUPPI) algorithm~\cite{Bertolini:2014bba} is used to mitigate the effect of pileup at the reconstructed particle level, making use of local shape information, event pileup properties and tracking information. Charged particles identified to be originating from pileup vertices are discarded. For each neutral particle, a local shape variable is computed using the surrounding charged particles compatible with the primary vertex within the tracker acceptance ($|\eta| < 2.5$), and using both charged and neutral particles in the region outside of the tracker coverage. The momenta of the neutral particles are then rescaled according to their probability to originate from the primary interaction vertex deduced from the local shape variable, superseding the need for jet-based pileup corrections~\cite{CMS-PAS-JME-16-003}. 


%To mitigate this effect, charged particles identified to be originating from pileup vertices are discarded and an offset correction is applied to correct for remaining contributions. Jet energy corrections are derived from simulation to bring measured response of jets to that of particle level jets on an average. In situ measurements of the momentum balance in dijet, $\text{photon} + \text{jet}$, $\PZ + \text{jet}$, and multijet events are used to account for any residual differences in jet energy scale in data and simulation~\cite{Khachatryan:2016kdb}. The jet energy resolution amounts typically to 15\% at 10\GeV, 8\% at 100\GeV, and 4\% at 1\TeV. Additional selection criteria are applied to each jet to remove jets potentially dominated by anomalous contributions from various subdetector components or reconstruction failures. 





\chapter{Measurement of the differential jet production cross section with respect to jet mass and transverse momentum in
Z + Jet events from pp collisions at $\sqrt{s}$ = 13 TeV}\label{chap:AN-18-240}

% copied from AN-18-240.tex

%\input{/Users/Om/Documents/thesis/AN-18-240/AN-18-240.tex}



%%%
%%% INTRO
%%%


\section{Introduction}

\input{AN-18-240/introductionThesis.tex}

%%%
%%%  DATA and MC 
%%%
\section{Data and MC}

\input{AN-18-240/dataandmc.tex}



%%%
%%%             TRIGGER
%%%
\section{Trigger}

The data are collected with single-lepton triggers, with muons (electrons) of $pt > 29 (37)$ GeV using the Isolated muon above 27 \GeV (Tight working point electron above 35 \GeV GSF $ || $ Photon above 200 \GeV) triggers, as recommended by the muon (egamma) POG. All triggers used for this analysis are prescaled to 1. 




%%%
%%%                        RECO and SELECTION
%%%

\section{Reconstruction and Selection}


\input{AN-18-240/recoandsel.tex}

\section{Data to MC Comparisons}


\input{AN-18-240/datatomccomp.tex}




\section{Detector Response}

\input{AN-18-240/detresp.tex}



%%%%%%%%%%%


% MASS RESPONSE MATRICES

\input{AN-18-240/responsematrix_u.tex}


\input{AN-18-240/responsematrix_sdB0.tex}

% RHO RESPONSE MATRICES
\input{AN-18-240/rhoresponsematrix_u.tex}


\input{AN-18-240/rhoresponsematrix_sdB0.tex}


%PURITY and STABILITY

\input{AN-18-240/purity_u.tex}

\input{AN-18-240/purity_sdB0.tex}




\section{Uncertainties}

\input{AN-18-240/uncert.tex}


\section{Unfolding}

\input{AN-18-240/tunfolding.tex}

\section{Results}

\input{AN-18-240/resultsThesis.tex}

\section{Summary}

In conclusion, we have presented a differential jet cross section measured in Z + Jet events in
bins of the ungroomed jet $\pt$ in conjunction with the ungroomed and groomed
jet mass (as well as dimensionless mass) using the ``soft drop'' (a.k.a. ``modified mass drop
tagger'') algorithm with 9 different combinations of parameters. 
The results are presented as the normalized cross section, normalized per reconstructed jet $\pt$ bin. 
Overall leading-order MC simulation agrees
reasonably well with the data within our uncertainties. 
Agreement below the Sudakov peak is slightly
worse than above. The application of a grooming algorithm
improves the overall precision, with larger improvement at low jet masses. 
This analysis improves over previous iterations by using various parameter values for the ``soft drop''
jet grooming algorithm, as well as by including an additional unfolding in both transverse momentum
and dimensionless mass, as was done by the ATLAS collaboration \cite{Aaboud:2017qwh} .





\appendix

\section{2016 data results}

This Appendix shows the distributions from the "Detector Response" through the "Results" sections of the main analysis note with only the 2016 data rather than the full Run 2 statistics seen in the main body of the note.



% MASS RESPONSE MATRICES

../../AN-18-240/responsematrix_u_2016.tex


../../../AN-18-240/responsematrix_sdB0_2016.tex

% RHO RESPONSE MATRICES
%\input{rhoresponsematrix_u_2016.tex}


%\input{rhoresponsematrix_sdB0_2016.tex}


%PURITY and STABILITY

../../AN-18-240/purity_u_2016.tex

../../../AN-18-240/purity_sdB0_2016.tex



../../../AN-18-240/uncert_2016.tex


../../../AN-18-240/tunfolding_2016.tex



../../AN-18-240/results_2016.tex



%\section{Data and MC comparisons for soft-dropped jet masses}
\section{2017 data results}

This Appendix shows the distributions from the "Detector Response" through the "Results" sections of the main analysis note with only the 2017 data rather than the full Run 2 statistics seen in the main body of the note.



% MASS RESPONSE MATRICES

../../AN-18-240/responsematrix_u_2017.tex


../../../AN-18-240/responsematrix_sdB0_2017.tex

% RHO RESPONSE MATRICES
%\input{rhoresponsematrix_u.tex}


%\input{rhoresponsematrix_sdB0.tex}


%PURITY and STABILITY

../../../AN-18-240/purity_u_2017.tex

../../../AN-18-240/purity_sdB0_2017.tex



../../AN-18-240/uncert_2017.tex


../../AN-18-240/tunfolding_2017.tex



../../../AN-18-240/results_2017.tex




\section{2018 data results}

This Appendix shows the distributions from the "Detector Response" through the "Results" sections of the main analysis note with only the 2018 data rather than the full Run 2 statistics seen in the main body of the note.



% MASS RESPONSE MATRICES

../../../AN-18-240/responsematrix_u_2018.tex


../../AN-18-240/responsematrix_sdB0_2018.tex

% RHO RESPONSE MATRICES
%\input{rhoresponsematrix_u_2016.tex}


%\input{rhoresponsematrix_sdB0_2016.tex}


%PURITY and STABILITY

../../../AN-18-240/purity_u_2018.tex

../../../AN-18-240/purity_sdB0_2018.tex



../../AN-18-240/uncert_2018.tex


../../AN-18-240/tunfolding_2018.tex



../../../AN-18-240/results_2018.tex





\chapter{Identification and Calibration of Boosted Hadronic W
Bosons within Fully Merged Top Quark Jets at 13 TeV}\label{chap:AN-17-177}



\clearpage

\chapter{Conclusion}\label{chap:conclusion}



Theoretical calculations are being prepared by our collegues and will soon be compared to the soft dropped jet unfolding results presented in this thesis.


%\center{The End.}
\include{Chapter-6/chap6}
%%%%%%%%%%%%%%%%%%%%%%%%%%%%%%%%%%%%%%%%%%%%%%%%%%%%%%%%%%%%%%
% Appendices
%
% Because of a quirk in LaTeX (see p. 48 of The LaTeX
% Companion, 2e), you cannot use \include along with
% \addtocontents if you want things to appear the proper
% sequence. Since the PSU Grad School requires
%%%%%%%%%%%%%%%%%%%%%%%%%%%%%%%%%%%%%%%%%%%%%%%%%%%%%%%%%%%%%%%
%\appendix
%

\chapter{Relativistic Kinematics}\label{chap:RelativisticKinematics}


\begin{figure}[htb]

\centering
\includegraphics[width=1.0\textwidth]{visuals/1to2splitting.png}
\caption{ A simple $ 1 \rightarrow 2  $ decay described by relativistic kinematics.}
\label{onetotwo}
\end{figure}

Consider the LO process depicted in ~\ref{onetotwo}, of a simple $ 1 \rightarrow 2  $ decay.


at LO there is a kinematic turn-off at $\pt R/2$ (where $R$ is the
distance parameter for the jet clustering), from the relativistic kinematics of a $1\rightarrow 2$ decay. 
However, for real jets the turn-off is closer to $\pt R/\sqrt{2}$ due to stochastic effects. 
To see this, consider a particle of energy $E$ and mass $m$ decaying to two massless
particles, each with an energy $E/2$ and separated by an angle $\theta$. 
The mass must satisfy $m^2 < \frac{E^2}{2}\left( 1 - \cos{\theta}\right)$. In the small
angle limit, this would be $m^2 < E^2\theta^2/4$, or $m < E\theta/2$.





According to relativistic kinematics the interaction can be described by the following equation:\newline

$p^{\mu} p_{\mu}  =  (p_1 + p_2)^{\mu} (p_1 + p_2)_{\mu}  $\newline

$p^{\mu} p_{\mu}  =  (p_1 + p_2)^{\mu} (p_1 + p_2)_{\mu}  $\newline

$m^2  = (E_1 + E_2)^2  - (\vec{p_1} + \vec{p_2} ) \dot (\vec{p_1} + \vec{p_2} )  $\newline


$m^2  \simeq 2 E_1 E_2 (1 - \cos \theta ) $\newline

If the energies of the splitting particles are equal then the equation simplifies since  $ E_1 = E_2 = \frac{E}{2} $ .


$m^2  < \frac{E^2}{2} ( 1 - \cos{\theta}  $\newline

$\frac{2m^2}{E^2}  < ( 1 - \cos{\theta}  $\newline

Using the small angle approximation this simplifies further.

$(1 - (1- \frac{\theta^2}{2}) )  \simeq  \frac{\theta^2}{2}  $\newline

Solving for mass, one finds :\newline

$m < E\theta/2$\newline

With more particle decays, the stochastic nature of the shower increases this to $m < E\theta/\sqrt{2}$.
Thus, a leading-order ($1\rightarrow 2$) decay will have a faster kinematic
turn-off than an all-orders ($1 \rightarrow$ many) decay.\newline

Solving for theta:\newline
 $\theta < \frac{2}{\gamma}$ where $\gamma$ is the lorentz factor $\gamma = \frac{1}{\sqrt{1-\frac{v^2}{c^2}}}  $.
%\chapter{Introduction}\label{chap:intro}

% \vspace{-5pt}
\section{Motivation}\label{sec:ch1:intro}

Within this dissertation, I provide a measurement of the differential jet cross section, as a function of the jet mass and transverse momentum, in events with a Z + Jet topology, with and without a jet grooming algorithm applied, using data collected by CMS experiment at LHC. The jet grooming used was the ``Soft Drop''
procedure 
%~\cite{softdrop}, wuth multiple values of the tunable parameters $\beta$ and $z_{cut}$. This represents the first, to my knowledge, measurement of it's kind with a light quark enriched jet sample at $\sqrt{s}$ = 13 TeV. 

Softdrop iteratively declusters a jet $j$ with distance parameter $R$ into two subjets, $j_1$ and $j_2$.
If the softdrop condition

\begin{equation}
  \frac{\min(p_{T1},p_{T2})}{p_{T1}+p_{T2}} > z_{cut} \cdot (\frac{\Delta R_{12}}{R})^\beta
\end{equation}

is met, then the procedure stops and $j$ is the final jet. Otherwise, the declustering continues - 
the higher $pt$ subjet is relabeled as $j$ and the lower $pt$ one is dropped.
By design, this condition fails for wide-angle soft radiation, which is therefore removed by the soft
drop procedure. The tunable parameters, $\beta$ and $z_{cut}$, control the degree of jet grooming:
$\beta$ tunes the algorithm's sensitivity to wide-angle radiation, while $z_{cut}$ sets the energy scale
of the grooming. In the case of $\beta \rightarrow \infty$, an ungroomed jet is returned. 
In the $\beta = 0$ case, the soft drop procedure is identical to the ``modified mass drop tagger'' (MMDT)
from Ref.
%~\cite{mmdt}. The soft drop algorithm removes soft and wide-angle radiation
from jets in a very theoretically controlled manner, making it suitable to separate
the ``hard'' and ``soft'' parts of the jet. Specifically, the soft drop
algorithm can remove non-global logarithms from correlations of
radiation within and between jets, which are extremely difficult to
compute theoretically
%~\cite{Dasgupta:2001sh,mmdt,softdrop,Dasgupta:2013via,Dasgupta:2015yua,Larkoski:2015zka}.

Comparing the production cross section for groomed and ungroomed jets separately allows us to
gain sensitivity to both the ``hard'' and ``soft'' jet physics. 
The groomed cross section can be directly compared to theoretical calculations of the jet mass
now and in the future, which is a very active area of theoretical research
at this time
%~\cite{Dasgupta:2012hg,Chien:2012ur,Jouttenus:2013hs,Almeida:2014uva,Liu:2014oog,Stewart:2014nna,Khelifa-Kerfa:2015mma,Frye:2016aiz,Kolodrubetz:2016dzb}. Furthermore, separating the hard and soft jet physics
allows a deeper understanding of the various effects involved in QCD
%radiation. In particular, Ref.~\cite{Frye:2016aiz} calculates the
groomed jet mass at next-to-next-to-leading order using soft colinear effective theory, matched to a
%parton shower at leading order using {\tt MCFM}~\cite{MCFM1,MCFM2}, and the authors of Ref.~\cite{mmdt} have
provided a next-to-leading logarithm calculation with traditional perturbative QCD, matched to a 
parton shower at leading order, also using {tt MCFM}.  We compare these theoretical predictions 
to our data in this paper for the first time in this channel at CMS. Both CMS and ATLAS have similar measurements in a dijet sample at %Ref.~\cite{cms_jetmassDijet, atlas_jetmass2}.

The analysis strategy is similar to that of %Ref.~\cite{cms_jetmassDijet}.
However, there are several differences. As in that paper, the cross section is also
unfolded in both jet mass and $pt$, however we also provide the measurement in jet $\rho$, dimensionless mass , and $pt$. While the previous measurement considered only one value for the soft drop parameter $\beta$, this analysis considers several.
We apply the soft drop algorithm to compare
directly to theoretical computations. Additionally, we not only measure the cross section as a function of mass, but also as a function of dimensionless mass, $\rho = 2log(m/(pt R))$, as is also done in the previously mentioned ATLAS measurement.  The dimensionless mass $\rho$ only weakly depends on $pt$, unlike mass, which is highly correlated. Additionally, the use of this variable aids in the separation of fixed order, perturbative and non-perturbative effects.
Finally, we also present the normalized differential cross section. We compute the cross sections normalized per $pt$ bin
(the ``normalized'' cross section) with respect to the jet $pt$ and jet mass 
by unfolding a binned two-dimensional distribution in $pt$ and mass
with widths $\Delta pt$ and $\Delta m$, respectively.

The normalized differential cross section in two forms :

\begin{equation}
\frac{1}{d\sigma/dpt}\frac{d^2\sigma}{dpt\,dm} = \frac{1}{N/\Delta pt} R(\frac{N_{ij}}{ \Delta pt \,\Delta m} )
\end{equation}


as well as :

\begin{equation}
\frac{1}{d\sigma/dpt}\frac{d^2\sigma}{dpt\,d\rho} = \frac{1}{N/\Delta pt} R(\frac{N_{ij}}{ \Delta pt \,\Delta \rho} )
\end{equation}


where $N$ is the total number of $Z+$jets events in our selection,
$N_{ij}$ is the number of such events in $pt$ bin $i$ and mass ($\rho$) bin $j$,
and $R(\alpha)$ is the unfolding procedure applied to the two-dimensional
distribution $\alpha$.

The 2 above normalized distributions are provided within for ungroomed  and groomed jets Anti-Kt Radious R$= 0.8$ jets. The groomed measurement is given in 9 configurations, one measurement is shown for jets groomed with every combination of 3 possible $\beta$ and $z_{cut}$ values (Where $\beta$ = 0 and $z_{cut}$ = 0 .1 is the current CMS default ):


$ \beta = [ 1,  0 , -1 ]  $

$ z_{cut}  = [ 0.15, 0.1, 0.05 ] $ 


These measurements currently represent humanity's highest energy measurement of a light quark enriched jet production cross section.


%%%%%%%%%%%%%%%%%%%%%%%%%%%%%%%%%%%%%%%%%%%%%%%%%%%%%%%%%%%%%%%
% ESM students need to include a Nontechnical Abstract as the %
% last appendix.                                              %
%%%%%%%%%%%%%%%%%%%%%%%%%%%%%%%%%%%%%%%%%%%%%%%%%%%%%%%%%%%%%%%
% This \include command should point to the file containing
% that abstract.
%\include{nontechnical-abstract}
%%%%%%%%%%%%%%%%%%%%%%%%%%%%%%%%%%%%%%%%%%%
} % End of the \allowdisplaybreak command %
%%%%%%%%%%%%%%%%%%%%%%%%%%%%%%%%%%%%%%%%%%%

%%%%%%%%%%%%%%%%
% BIBLIOGRAPHY %
%%%%%%%%%%%%%%%%
% You can use BibTeX or other bibliography facility for your
% bibliography. LaTeX's standard stuff is shown below. If you
% bibtex, then this section should look something like:
   \begin{singlespace}
   %\bibliographystyle{plainnat}
   %\phantomsection \addcontentsline{toc}{chapter}{Bibliography}
   \bibliography{AN-18-240} % was ref 

   %\printbibliography[heading=bibintoc]
   \end{singlespace}
%\begin{singlespace}
%\begin{thebibliography}{99}
%\addcontentsline{toc}{chapter}{Bibliography}
%\frenchspacing

%\bibitem{Wisdom87} J. Wisdom, ``Rotational Dynamics of Irregularly Shaped Natural Satellites,'' \emph{The Astronomical Journal}, Vol.~94, No.~5, 1987  pp. 1350--1360.

%\bibitem{G&H83} J. Guckenheimer and P. Holmes, \emph{Nonlinear Oscillations, Dynamical Systems, and Bifurcations of Vector Fields}, Springer-Verlag, New York, 1983.

%\end{thebibliography}
%\end{singlespace}

\backmatter

% Vita
% \vita{SupplementaryMaterial/Vita}

\end{document}

