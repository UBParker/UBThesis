
\chapter{Introduction}

This thesis presents a measurement of the double differential jet production cross section as a function of the jet mass and transverse momentum, in events with a Z + Jet topology, using 136 $fb^{-1}$ of data acquired from the Compact Muon Solenoid (CMS) experiment at the Large Hadron Collider (LHC) during Run 2 of data taking. This measurement improves upon the previous measurement by CMS experiment at 7 TeV center-of-mass energy ~\cite{Chatrchyan:2013vbb} and the analogous dijet measurement at 13 TeV ~\cite{Sirunyan:2018xdh} . There is a similar measurement by ATLAS experiment in the dijet channel ~\cite{Aaboud:2017qwh} however this is the first measurement in the Z + Jets channel to be presented at 13 TeV. %A leading source of systematic uncertainty in the measurement is the jet mass scale which can be improved upon using studies such as the one in chapter \ref{chap:Wtag}


The standard model of particle physics, while describing our universe well on many scales, has yet to be precisely measured in all energy regimes. Recent theoretical advances in higher order QCD calculations have provided a way to compare the standard model's predictions to precision measurements of data and monte carlo simulation.  Studying Z + jet events will yield a light quark enriched jet sample, which has not yet been studied at $\sqrt{s}$ = 13 TeV. Comparing groomed and ungroomed jets will allow us the better understand the jet mass in all energy regimes since the groomed jets will have varying amounts of soft and collinear radiation with respect to the ungroomed counterpart.For ungroomed jets, leading-order and next-to-leading order QCD Monte Carlo programs are found to predict the jet mass spectrum in the data reasonably well, with some disagreement at very low and very high masses. For groomed jets, the agreement between the Monte Carlo programs and the data improves overall, and extends lower in jet mass due to the removal of soft and colinear portions of the jet.First-principles theoretical calculations of the groomed jet mass are also compared to the data, and agree with the data within the range of acceptability of the calculations. Ultimately these measurements will be used to tune Monte Carlo generators, producing more accurate parton showering simulations, leading to tighter constraint of backgrounds in future searches for new physics.